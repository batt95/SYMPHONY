\batchmode


\documentclass[twoside,11pt]{book}
\RequirePackage{ifthen}




\oddsidemargin 0in
\evensidemargin 0in
\topmargin 0in
\textwidth 6.5in

\newlength{\headwidth} 

\setlength{\headwidth}{\textwidth} 

\addtolength{\headwidth}{-4.4444pt} 
\textheight 8.7in
\setcounter{secnumdepth}{3}


\catcode`\@=11\relax   %allow @ in macro names


\if@twoside         % If two-sided printing.
\else               % If one-sided printing.

\fi


\catcode`\@=12\relax   %disable @ in macro names


\pagestyle{headings}


\markright{\bf\thesection }{}


\usepackage{epsfig,html}
\usepackage{amsmath}
\usepackage{amssymb}
\usepackage{fullpage}
\usepackage{easyeqn}
\usepackage{fancyvrb}
\usepackage{listings}
\usepackage{url}
\usepackage{splitidx}
\usepackage[usenames]{color}


\makeindex
\newindex[Index of C API Functions]{f}
\newindex[Index of User Callback API Functions]{cf}
\newindex[Index of Parameters]{p}

\lstloadlanguages{C++}

%
\providecommand{\be}{\begin{enumerate}}%
\providecommand{\ee}{\end{enumerate}}%
\providecommand{\bc}{\begin{center}}%
\providecommand{\ec}{\end{center}}%
\providecommand{\bt}{\begin{tabular}}%
\providecommand{\et}{\end{tabular}}%
\providecommand{\bd}{\begin{description}}%
\providecommand{\ed}{\end{description}}%
\providecommand{\bi}{\begin{itemize}}%
\providecommand{\ei}{\end{itemize}}%
\providecommand{\bv}{\begin{verbatim}
}
\newcommand{\ev}{\end{verbatim}
}%
\providecommand{\functiondef}[1]{\subsubsection{#1}}%
\providecommand{\firstfuncdef}[1]{\subsubsection{#1}}%
\providecommand{\code}[1]{{\color{Brown}\texttt{#1}}}%
\providecommand{\describe}{\item[{\color{cyan}Description:}] \hfill}%
\providecommand{\args}{\item[{\color{Brown}Arguments:}] \hfill}%
\providecommand{\returns}{\item[{\color{green}Return values:}] \hfill}%
\providecommand{\postp}{\item[Post-processing:] \hfill}%
\providecommand{\nopostp}{\item[Post-processing:] None \hfill} 

%
\providecommand{\BB}{{\sc SYMPHONY}}%
\providecommand{\TM}{{\sc TreeManager}}%
\providecommand{\LP}{{\sc LP}}%
\providecommand{\ra}{$\rightarrow$}%
\providecommand{\vmin}{\mathop{\text{vmin}}}%
\providecommand{\Z}{\mathbb{Z}}%
\providecommand{\Q}{\mathbb{Q}} 
%
\renewcommand{\Re}{\mathbb{R}}%
\providecommand{\ptt}[1]{{\tt {\color{BrickRed} #1}}}%
\providecommand{\bs}{{$\backslash$}} 


\usepackage[latex2html,dvipdfm,colorlinks]{hyperref}
\hypersetup{
  linkcolor=blue,
}



\setlength{\parindent}{0in} 

\setlength{\parskip}{0.1in} 





\pagecolor[gray]{.7}

\usepackage[latin1]{inputenc}



\makeatletter

\makeatletter
\count@=\the\catcode`\_ \catcode`\_=8 
\newenvironment{tex2html_wrap}{}{}%
\catcode`\<=12\catcode`\_=\count@
\newcommand{\providedcommand}[1]{\expandafter\providecommand\csname #1\endcsname}%
\newcommand{\renewedcommand}[1]{\expandafter\providecommand\csname #1\endcsname{}%
  \expandafter\renewcommand\csname #1\endcsname}%
\newcommand{\newedenvironment}[1]{\newenvironment{#1}{}{}\renewenvironment{#1}}%
\let\newedcommand\renewedcommand
\let\renewedenvironment\newedenvironment
\makeatother
\let\mathon=$
\let\mathoff=$
\ifx\AtBeginDocument\undefined \newcommand{\AtBeginDocument}[1]{}\fi
\newbox\sizebox
\setlength{\hoffset}{0pt}\setlength{\voffset}{0pt}
\addtolength{\textheight}{\footskip}\setlength{\footskip}{0pt}
\addtolength{\textheight}{\topmargin}\setlength{\topmargin}{0pt}
\addtolength{\textheight}{\headheight}\setlength{\headheight}{0pt}
\addtolength{\textheight}{\headsep}\setlength{\headsep}{0pt}
\setlength{\textwidth}{451pt}
\setlength{\textheight}{554pt}
\newwrite\lthtmlwrite
\makeatletter
\let\realnormalsize=\normalsize
\global\topskip=2sp
\def\preveqno{}\let\real@float=\@float \let\realend@float=\end@float
\def\@float{\let\@savefreelist\@freelist\real@float}
\def\liih@math{\ifmmode$\else\bad@math\fi}
\def\end@float{\realend@float\global\let\@freelist\@savefreelist}
\let\real@dbflt=\@dbflt \let\end@dblfloat=\end@float
\let\@largefloatcheck=\relax
\let\if@boxedmulticols=\iftrue
\def\@dbflt{\let\@savefreelist\@freelist\real@dbflt}
\def\adjustnormalsize{\def\normalsize{\mathsurround=0pt \realnormalsize
 \parindent=0pt\abovedisplayskip=0pt\belowdisplayskip=0pt}%
 \def\phantompar{\csname par\endcsname}\normalsize}%
\def\lthtmltypeout#1{{\let\protect\string \immediate\write\lthtmlwrite{#1}}}%
\newcommand\lthtmlhboxmathA{\adjustnormalsize\setbox\sizebox=\hbox\bgroup\kern.05em }%
\newcommand\lthtmlhboxmathB{\adjustnormalsize\setbox\sizebox=\hbox to\hsize\bgroup\hfill }%
\newcommand\lthtmlvboxmathA{\adjustnormalsize\setbox\sizebox=\vbox\bgroup %
 \let\ifinner=\iffalse \let\)\liih@math }%
\newcommand\lthtmlboxmathZ{\@next\next\@currlist{}{\def\next{\voidb@x}}%
 \expandafter\box\next\egroup}%
\newcommand\lthtmlmathtype[1]{\gdef\lthtmlmathenv{#1}}%
\newcommand\lthtmllogmath{\dimen0\ht\sizebox \advance\dimen0\dp\sizebox
  \ifdim\dimen0>.95\vsize
   \lthtmltypeout{%
*** image for \lthtmlmathenv\space is too tall at \the\dimen0, reducing to .95 vsize ***}%
   \ht\sizebox.95\vsize \dp\sizebox\z@ \fi
  \lthtmltypeout{l2hSize %
:\lthtmlmathenv:\the\ht\sizebox::\the\dp\sizebox::\the\wd\sizebox.\preveqno}}%
\newcommand\lthtmlfigureA[1]{\let\@savefreelist\@freelist
       \lthtmlmathtype{#1}\lthtmlvboxmathA}%
\newcommand\lthtmlpictureA{\bgroup\catcode`\_=8 \lthtmlpictureB}%
\newcommand\lthtmlpictureB[1]{\lthtmlmathtype{#1}\egroup
       \let\@savefreelist\@freelist \lthtmlhboxmathB}%
\newcommand\lthtmlpictureZ[1]{\hfill\lthtmlfigureZ}%
\newcommand\lthtmlfigureZ{\lthtmlboxmathZ\lthtmllogmath\copy\sizebox
       \global\let\@freelist\@savefreelist}%
\newcommand\lthtmldisplayA{\bgroup\catcode`\_=8 \lthtmldisplayAi}%
\newcommand\lthtmldisplayAi[1]{\lthtmlmathtype{#1}\egroup\lthtmlvboxmathA}%
\newcommand\lthtmldisplayB[1]{\edef\preveqno{(\theequation)}%
  \lthtmldisplayA{#1}\let\@eqnnum\relax}%
\newcommand\lthtmldisplayZ{\lthtmlboxmathZ\lthtmllogmath\lthtmlsetmath}%
\newcommand\lthtmlinlinemathA{\bgroup\catcode`\_=8 \lthtmlinlinemathB}
\newcommand\lthtmlinlinemathB[1]{\lthtmlmathtype{#1}\egroup\lthtmlhboxmathA
  \vrule height1.5ex width0pt }%
\newcommand\lthtmlinlineA{\bgroup\catcode`\_=8 \lthtmlinlineB}%
\newcommand\lthtmlinlineB[1]{\lthtmlmathtype{#1}\egroup\lthtmlhboxmathA}%
\newcommand\lthtmlinlineZ{\egroup\expandafter\ifdim\dp\sizebox>0pt %
  \expandafter\centerinlinemath\fi\lthtmllogmath\lthtmlsetinline}
\newcommand\lthtmlinlinemathZ{\egroup\expandafter\ifdim\dp\sizebox>0pt %
  \expandafter\centerinlinemath\fi\lthtmllogmath\lthtmlsetmath}
\newcommand\lthtmlindisplaymathZ{\egroup %
  \centerinlinemath\lthtmllogmath\lthtmlsetmath}
\def\lthtmlsetinline{\hbox{\vrule width.1em \vtop{\vbox{%
  \kern.1em\copy\sizebox}\ifdim\dp\sizebox>0pt\kern.1em\else\kern.3pt\fi
  \ifdim\hsize>\wd\sizebox \hrule depth1pt\fi}}}
\def\lthtmlsetmath{\hbox{\vrule width.1em\kern-.05em\vtop{\vbox{%
  \kern.1em\kern0.8 pt\hbox{\hglue.17em\copy\sizebox\hglue0.8 pt}}\kern.3pt%
  \ifdim\dp\sizebox>0pt\kern.1em\fi \kern0.8 pt%
  \ifdim\hsize>\wd\sizebox \hrule depth1pt\fi}}}
\def\centerinlinemath{%
  \dimen1=\ifdim\ht\sizebox<\dp\sizebox \dp\sizebox\else\ht\sizebox\fi
  \advance\dimen1by.5pt \vrule width0pt height\dimen1 depth\dimen1 
 \dp\sizebox=\dimen1\ht\sizebox=\dimen1\relax}

\def\lthtmlcheckvsize{\ifdim\ht\sizebox<\vsize 
  \ifdim\wd\sizebox<\hsize\expandafter\hfill\fi \expandafter\vfill
  \else\expandafter\vss\fi}%
\providecommand{\selectlanguage}[1]{}%
\makeatletter \tracingstats = 1 
\providecommand{\Beta}{\textrm{B}}
\providecommand{\Mu}{\textrm{M}}
\providecommand{\Kappa}{\textrm{K}}
\providecommand{\Rho}{\textrm{R}}
\providecommand{\Epsilon}{\textrm{E}}
\providecommand{\Chi}{\textrm{X}}
\providecommand{\Iota}{\textrm{J}}
\providecommand{\omicron}{\textrm{o}}
\providecommand{\Zeta}{\textrm{Z}}
\providecommand{\Eta}{\textrm{H}}
\providecommand{\Nu}{\textrm{N}}
\providecommand{\Omicron}{\textrm{O}}
\providecommand{\Tau}{\textrm{T}}
\providecommand{\Alpha}{\textrm{A}}


\begin{document}
\pagestyle{empty}\thispagestyle{empty}\lthtmltypeout{}%
\lthtmltypeout{latex2htmlLength hsize=\the\hsize}\lthtmltypeout{}%
\lthtmltypeout{latex2htmlLength vsize=\the\vsize}\lthtmltypeout{}%
\lthtmltypeout{latex2htmlLength hoffset=\the\hoffset}\lthtmltypeout{}%
\lthtmltypeout{latex2htmlLength voffset=\the\voffset}\lthtmltypeout{}%
\lthtmltypeout{latex2htmlLength topmargin=\the\topmargin}\lthtmltypeout{}%
\lthtmltypeout{latex2htmlLength topskip=\the\topskip}\lthtmltypeout{}%
\lthtmltypeout{latex2htmlLength headheight=\the\headheight}\lthtmltypeout{}%
\lthtmltypeout{latex2htmlLength headsep=\the\headsep}\lthtmltypeout{}%
\lthtmltypeout{latex2htmlLength parskip=\the\parskip}\lthtmltypeout{}%
\lthtmltypeout{latex2htmlLength oddsidemargin=\the\oddsidemargin}\lthtmltypeout{}%
\makeatletter
\if@twoside\lthtmltypeout{latex2htmlLength evensidemargin=\the\evensidemargin}%
\else\lthtmltypeout{latex2htmlLength evensidemargin=\the\oddsidemargin}\fi%
\lthtmltypeout{}%
\makeatother
\setcounter{page}{1}
\onecolumn

% !!! IMAGES START HERE !!!

\setcounter{secnumdepth}{3}
\stepcounter{chapter}
\stepcounter{section}
\stepcounter{section}
\stepcounter{section}
\stepcounter{section}
\stepcounter{section}
\stepcounter{section}
\stepcounter{chapter}
\stepcounter{section}
\stepcounter{subsection}
\stepcounter{subsection}
\stepcounter{section}
\stepcounter{subsection}
\stepcounter{subsubsection}
\stepcounter{subsubsection}
\stepcounter{subsubsection}
\stepcounter{subsubsection}
\stepcounter{subsection}
\stepcounter{subsubsection}
\stepcounter{subsubsection}
{\newpage\clearpage
\lthtmlfigureA{figure641}%
\begin{figure}\begin{tabular}{ll}
\hline
\texttt{--enable-debug} & compile all projects with debug options set \\
\texttt{--enable-debug-symphony} & compile only SYMPHONY project with debug options \\
\texttt{--enable-doscompile} & Under Cygwin, compile so that executables do
not depend on the CYGWIN DLL \\
\texttt{--enable-static} & build static libraries \\
\texttt{--enable-static-executable} &  create a complete static executable \\
\texttt{--enable-gnu-packages} & compile with GNU packages \\
& compile interactive optimizer with readline library \\
\hline
\texttt{--disable-cgl-cuts} & disable generic cut generation \\
\texttt{--enable-sensitivity-analysis} & compile in the sensitivity analysis features \\
\texttt{--enable-root-only} & process only the root node \\
\texttt{--enable-frac-branching} & compile in the fractional branching option \\
\texttt{--enable-tests}&  perform additional sanity checks (for debugging purposes) \\
\texttt{--enable-tm-tests }& perform more tests  \\
\texttt{--enable-trace-path}&  additional debugging options \\
\texttt{--enable-cut-check}& additional debugging options \\
\texttt{--enable-statistics}& additional statistics \\
\texttt{--enable-pseudo-costs}& enable some experimental pseudo-cost branching tools \\
\texttt{--enable-draw-graph} &  enable IGD graph drawing application \\
\hline
\texttt{ --with-XXX-incdir} &  specify the directory with the header files for the XXX package \\
&where XXX is one of LP solver packages: cplex, glpk, osl, soplex, \\
& xpress \\
\texttt{--with-XXX-lib} &  specify the flags to link with the library  
XXX package \\
&where XXX is one of LP solver packages: cplex, glpk, osl, soplex, \\
& xpress \\
\texttt{--with-lp-solver[=lpsolver]} &  specify the LP solver in small 
letters (default lpsolver=clp) \\
\texttt{--with-application} &  compile the application library \\
\hline
\texttt{--enable-openmp} &   compile in OpenMP features \\
\texttt{--with-pvm } &  compile in parallel architecture (assuming that pvm is \\
&installed and the variable PVM\_ROOT is defined.) \\
\texttt{--without-cg} &  compile without cut generator module \\
\texttt{--without-cp} &  compile without cut pool module \\
\texttt{--without-lp} &  compile without LP solver module \\
\texttt{--without-tm} &  compile without tree manager module
\end{tabular}

\end{figure}%
\lthtmlfigureZ
\lthtmlcheckvsize\clearpage}

\stepcounter{subsubsection}
\stepcounter{subsubsection}
\stepcounter{paragraph}
\stepcounter{paragraph}
\stepcounter{subsubsection}
\stepcounter{subsection}
\stepcounter{subsubsection}
{\newpage\clearpage
\lthtmlinlinemathA{tex2html_wrap_inline21951}%
\bgroup\color{Brown}$ \backslash$\egroup%
\lthtmlinlinemathZ
\lthtmlcheckvsize\clearpage}

\stepcounter{subsubsection}
\stepcounter{subsubsection}
\stepcounter{subsubsection}
\stepcounter{subsection}
\stepcounter{subsubsection}
\stepcounter{subsubsection}
\stepcounter{subsubsection}
\stepcounter{chapter}
\stepcounter{section}
\stepcounter{subsection}
\stepcounter{subsection}
\stepcounter{subsection}
\stepcounter{subsection}
\stepcounter{subsection}
\stepcounter{subsection}
\stepcounter{section}
\stepcounter{section}
\stepcounter{subsection}
\stepcounter{paragraph}
\stepcounter{paragraph}
\stepcounter{paragraph}
\stepcounter{paragraph}
\stepcounter{paragraph}
\stepcounter{paragraph}
\stepcounter{paragraph}
\stepcounter{paragraph}
{\newpage\clearpage
\lthtmlfigureA{figure2175}%
\begin{figure}{\color{Brown}
\begin{Verbatim}
[frame=lines]
int main(int argc, char **argv)
{
   sym_environment *env = sym_open_environment();
   sym_parse_command_line(env, argc, argv);
   sym_load_problem(env);
   sym_solve(env);
   sym_close_environment(env);
}\end{Verbatim}

}

\end{figure}%
\lthtmlfigureZ
\lthtmlcheckvsize\clearpage}

{\newpage\clearpage
\lthtmlfigureA{figure2200}%
\begin{figure}{\color{Brown}
\begin{Verbatim}
[frame=lines]
int main(int argc, char **argv)
{
   sym_environment *env = sym_open_environment();
   sym_parse_command_line(env, argc, argv);
   sym_load_problem(env);
   sym_set_int_param(env, "find_first_feasible", TRUE);
   sym_set_int_param(env, "node_selection_strategy", DEPTH_FIRST_SEARCH);
   sym_solve(env);
   sym_set_int_param(env, "find_first_feasible", FALSE);
   sym_set_int_param(env, "node_selection_strategy", BEST_FIRST_SEARCH);
   sym_warm_solve(env);
}\end{Verbatim}

}

\end{figure}%
\lthtmlfigureZ
\lthtmlcheckvsize\clearpage}

{\newpage\clearpage
\lthtmlfigureA{figure2207}%
\begin{figure}{\color{Brown}
\begin{Verbatim}
[frame=lines]
int main(int argc, char **argv)
{
   warm_start_desc *ws;
   sym_environment *env = sym_open_environment();
   sym_parse_command_line(env, argc, argv);
   sym_load_problem(env);
   sym_set_int_param(env, "node_limit", 100);
   sym_set_int_param(env, "keep_warm_start", TRUE);
   sym_solve(env);
   ws = sym_get_warm_start(env);
   sym_set_int_param(env, "node_limit", -1);
   sym_warm_solve(env);
   sym_set_obj_coeff(env, 0, 100);
   sym_set_obj_coeff(env, 200, 150);
   sym_set_warm_start(ws);
   sym_warm_solve(env);
}\end{Verbatim}

}

\end{figure}%
\lthtmlfigureZ
\lthtmlcheckvsize\clearpage}

{\newpage\clearpage
\lthtmlinlinemathA{tex2html_wrap_inline22022}%
\bgroup\color{Brown}$ i$\egroup%
\lthtmlinlinemathZ
\lthtmlcheckvsize\clearpage}

{\newpage\clearpage
\lthtmlinlinemathA{tex2html_wrap_inline22024}%
\bgroup\color{Brown}$ i^\textrm{th}$\egroup%
\lthtmlinlinemathZ
\lthtmlcheckvsize\clearpage}

{\newpage\clearpage
\lthtmlfigureA{figure2215}%
\begin{figure}{\color{Brown}
\begin{Verbatim}
[frame=lines]
int main(int argc, char **argv)
{
   sym_environment *env = sym_open_environment();
   sym_parse_command_line(env, argc, argv);
   sym_load_problem(env);
   sym_set_obj2_coeff(env, 0, 1);
   sym_mc_solve(env);
}\end{Verbatim}

}

\end{figure}%
\lthtmlfigureZ
\lthtmlcheckvsize\clearpage}

\stepcounter{subsection}
{\newpage\clearpage
\lthtmlfigureA{figure2231}%
\begin{figure}{\color{Brown}
\begin{Verbatim}
[frame=lines]
int main(int argc, char **argv)
{
   OsiSymSolverInterface si;
   si.parseCommandLine(argc, argv);
   si.loadProblem();
   si.branchAndBound();
}\end{Verbatim}

}

\end{figure}%
\lthtmlfigureZ
\lthtmlcheckvsize\clearpage}

\stepcounter{subsection}
\stepcounter{section}
\stepcounter{chapter}
\stepcounter{section}
{\newpage\clearpage
\lthtmlinlinemathA{tex2html_wrap_inline22037}%
\bgroup\color{Brown}$ S$\egroup%
\lthtmlinlinemathZ
\lthtmlcheckvsize\clearpage}

{\newpage\clearpage
\lthtmlinlinemathA{tex2html_wrap_inline22039}%
\bgroup\color{Brown}$ c \in {\bf R}^S$\egroup%
\lthtmlinlinemathZ
\lthtmlcheckvsize\clearpage}

{\newpage\clearpage
\lthtmlinlinemathA{tex2html_wrap_inline22041}%
\bgroup\color{Brown}$ \hat{s} \in S$\egroup%
\lthtmlinlinemathZ
\lthtmlcheckvsize\clearpage}

{\newpage\clearpage
\lthtmlinlinemathA{tex2html_wrap_inline22049}%
\bgroup\color{Brown}$ \hat{s}$\egroup%
\lthtmlinlinemathZ
\lthtmlcheckvsize\clearpage}

{\newpage\clearpage
\lthtmlinlinemathA{tex2html_wrap_inline22053}%
\bgroup\color{Brown}$ n$\egroup%
\lthtmlinlinemathZ
\lthtmlcheckvsize\clearpage}

{\newpage\clearpage
\lthtmlinlinemathA{tex2html_wrap_inline22057}%
\bgroup\color{Brown}$ S_1, \ldots, S_n$\egroup%
\lthtmlinlinemathZ
\lthtmlcheckvsize\clearpage}

{\newpage\clearpage
\lthtmlinlinemathA{tex2html_wrap_inline22059}%
\bgroup\color{Brown}$ \cup_{i = 1}^n S_i = S$\egroup%
\lthtmlinlinemathZ
\lthtmlcheckvsize\clearpage}

\stepcounter{section}
{\newpage\clearpage
\lthtmlpictureA{tex2html_wrap3363}%
\framebox[6.5in]{
\begin{minipage}{6.0in}
\vskip .1in
{\rm
{\bf Bounding Operation}\\
\underbar{Input:} A subproblem ${\cal S}$, described in
terms of a ``small'' set of inequalities ${\cal L'}$\  such that ${\cal
S} = \{x^s : s \in {\cal F}\;\hbox{\rm and}\;ax^s \leq \beta\;\forall
\;(a,\beta) \in {\cal L'}\}$\  and $\alpha$, an upper bound on the global 
optimal value. \\
\underbar{Output:} Either (1) an optimal solution $s^* \in {\cal S}$\  to
the subproblem, (2) a lower bound on the optimal value of the 
subproblem, or (3) a message {\tt pruned} indicating that the
subproblem should not be considered further. \\
{\bf Step 1.} Set ${\cal C} \leftarrow {\cal L'}$. \\
{\bf Step 2.} Solve the LP $\min\{cx : ax \leq \beta\;\forall\;(a, \beta) 
\in {\cal C}\}$. \\
{\bf Step 3.} If the LP has a feasible solution $\hat{x}$, then go to
Step 4. Otherwise, STOP and output {\tt pruned}. This subproblem has no 
feasible solutions. \\
{\bf Step 4.} If $c\hat{x} < \alpha$, then go to Step
5. Otherwise, STOP and output {\tt pruned}. This subproblem
cannot produce a solution of value better than $\alpha$. \\
{\bf Step 5.} If $\hat{x}$\  is the incidence vector of some $\hat{s}
\in {\cal S}$, then $\hat{s}$\  is the optimal solution to this
subproblem. STOP and output $\hat{s}$\  as $s^*$. Otherwise, apply
separation algorithms and heuristics to $\hat{x}$\  to get a set of
violated inequalities ${\cal C'}$. If ${\cal C'} = \emptyset$, then
$c\hat{x}$\  is a lower bound on the value of an optimal element of
${\cal S}$.  STOP and return $\hat{x}$\  and the lower bound
$c\hat{x}$. Otherwise, set ${\cal C} \leftarrow {\cal C} \cup {\cal
C'}$\  and go to Step 2.}
\end{minipage}
}%
\lthtmlpictureZ
\lthtmlcheckvsize\clearpage}

{\newpage\clearpage
\lthtmlinlinemathA{tex2html_wrap_inline22076}%
\bgroup\color{Brown}$ \hbox{\em CP} =
(E, {\cal F})$\egroup%
\lthtmlinlinemathZ
\lthtmlcheckvsize\clearpage}

{\newpage\clearpage
\lthtmlinlinemathA{tex2html_wrap_inline22078}%
\bgroup\color{Brown}$ E$\egroup%
\lthtmlinlinemathZ
\lthtmlcheckvsize\clearpage}

{\newpage\clearpage
\lthtmlinlinemathA{tex2html_wrap_inline22080}%
\bgroup\color{Brown}$ {\cal F}
\subseteq 2^E$\egroup%
\lthtmlinlinemathZ
\lthtmlcheckvsize\clearpage}

{\newpage\clearpage
\lthtmlinlinemathA{tex2html_wrap_inline22082}%
\bgroup\color{Brown}$ c \in {\bf R}^E$\egroup%
\lthtmlinlinemathZ
\lthtmlcheckvsize\clearpage}

{\newpage\clearpage
\lthtmlinlinemathA{tex2html_wrap_inline22084}%
\bgroup\color{Brown}$ {\cal F}$\egroup%
\lthtmlinlinemathZ
\lthtmlcheckvsize\clearpage}

{\newpage\clearpage
\lthtmlinlinemathA{tex2html_wrap_inline22086}%
\bgroup\color{Brown}$ {\cal
P}$\egroup%
\lthtmlinlinemathZ
\lthtmlcheckvsize\clearpage}

{\newpage\clearpage
\lthtmlinlinemathA{tex2html_wrap_inline22090}%
\bgroup\color{Brown}$ {\cal L}$\egroup%
\lthtmlinlinemathZ
\lthtmlcheckvsize\clearpage}

{\newpage\clearpage
\lthtmlinlinemathA{tex2html_wrap_indisplay22094}%
$\displaystyle {\cal P} = \{x \in {\bf R}^n: ax \leq \beta\;\;\forall\;(a, \beta) \in  {\cal L}\}.$%
\lthtmlindisplaymathZ
\lthtmlcheckvsize\clearpage}

{\newpage\clearpage
\lthtmlpictureA{tex2html_wrap3377}%
\framebox[6.5in]{
\begin{minipage}{6.0in}
\vskip .1in
{\rm
{\bf Branching Operation} \\
\underbar{Input:} A subproblem ${\cal S}$\  and $\hat{x}$, the LP solution
yielding the lower bound. \\
\underbar{Output:} $S_1, \ldots, S_p$\  such that ${\cal S} = \cup_{i = 1}^p
S_i$. \\
{\bf Step 1.} Determine sets ${\cal L}_1, \ldots, {\cal L}_p$\  of
inequalities such that ${\cal S} = \cup_{i = 1}^n \{x \in {\cal S}: ax \leq
\beta\;\forall\;(a, \beta) \in {\cal L}_i\}$\  and $\hat{x} \notin
\cup_{i = 1}^n S_i$. \\
{\bf Step 2.} Set $S_i = \{x \in {\cal S}: ax \leq
\beta\;\;\forall\;(a, \beta) \in {\cal L}_i \cup {\cal L}'\}$\  where 
${\cal L'}$\  is the set of inequalities used to describe ${\cal S}$.}
\end{minipage}
}%
\lthtmlpictureZ
\lthtmlcheckvsize\clearpage}

{\newpage\clearpage
\lthtmlpictureA{tex2html_wrap3383}%
\framebox[6.5in]{
\begin{minipage}{6.0in}
\vskip .1in
{\rm
{\bf Generic Branch and Cut Algorithm}\\
\underbar{Input:} A data array specifying the problem instance.\\
\underbar{Output:} The global optimal solution $s^*$\  to the problem
instance. \\
{\bf Step 1.} Generate a ``good'' feasible solution ${\hat s}$\  using 
heuristics. Set $\alpha \leftarrow c(\hat{s})$. \\
{\bf Step 2.} Generate the first subproblem ${\cal S}^I$\  by constructing a
small set ${\cal L'}$\  of inequalities valid for ${\cal P}$. Set $A
\leftarrow \{{\cal S}^I\}$. \\
{\bf Step 3.} If $A = \emptyset$, STOP and output $\hat{s}$\  as the
global optimum $s^*$. Otherwise, choose some ${\cal S} \in A$. Set $A
\leftarrow A \setminus \{{\cal S}\}$. Process ${\cal S}$. \\
{\bf Step 4.} If the result of Step 3 is a feasible solution
$\overline{s}$, then $c\overline{s} < c\hat{s}$.
Set $\hat{s} \leftarrow \overline{s}$\  and $\alpha \leftarrow 
c(\overline{s})$\  and go to Step 3. If the subproblem was pruned, go to
Step 3. Otherwise, go to Step 5. \\
{\bf Step 5.} Perform the branching operation. Add the set of
subproblems generated to $A$\  and go to Step 3.}
\end{minipage}
}%
\lthtmlpictureZ
\lthtmlcheckvsize\clearpage}

\stepcounter{section}
\stepcounter{subsection}
\stepcounter{subsection}
\stepcounter{subsubsection}
\stepcounter{subsubsection}
\stepcounter{subsubsection}
\stepcounter{subsection}
{\newpage\clearpage
\lthtmlfigureA{figure2963}%
\begin{figure}\centering
\psfig{figure=pbandc.eps}

\end{figure}%
\lthtmlfigureZ
\lthtmlcheckvsize\clearpage}

\stepcounter{subsubsection}
\stepcounter{subsubsection}
\stepcounter{subsubsection}
\stepcounter{subsubsection}
\stepcounter{subsubsection}
\stepcounter{subsection}
\stepcounter{section}
\stepcounter{subsection}
\stepcounter{subsubsection}
\stepcounter{subsubsection}
\stepcounter{paragraph}
\stepcounter{paragraph}
\stepcounter{subsubsection}
{\newpage\clearpage
\lthtmldisplayA{displaymath22133}%
\begin{displaymath}\begin{array}{lrcl} \mathop{\text{vmin}}& [cx, dx],\\\textrm{s.t.} & Ax & \leq & b, \\& x & \in & \mathbb{Z}^{n}. \end{array}\end{displaymath}%
\lthtmldisplayZ
\lthtmlcheckvsize\clearpage}

{\newpage\clearpage
\lthtmlinlinemathA{tex2html_wrap_inline22135}%
\bgroup\color{Brown}$ p$\egroup%
\lthtmlinlinemathZ
\lthtmlcheckvsize\clearpage}

{\newpage\clearpage
\lthtmlinlinemathA{tex2html_wrap_inline22137}%
\bgroup\color{Brown}$ q$\egroup%
\lthtmlinlinemathZ
\lthtmlcheckvsize\clearpage}

{\newpage\clearpage
\lthtmlinlinemathA{tex2html_wrap_inline22139}%
\bgroup\color{Brown}$ cq \leq cp$\egroup%
\lthtmlinlinemathZ
\lthtmlcheckvsize\clearpage}

{\newpage\clearpage
\lthtmlinlinemathA{tex2html_wrap_inline22141}%
\bgroup\color{Brown}$ dq \leq dp$\egroup%
\lthtmlinlinemathZ
\lthtmlcheckvsize\clearpage}

{\newpage\clearpage
\lthtmlinlinemathA{tex2html_wrap_inline22143}%
\bgroup\color{Brown}$ 0 \leq \alpha \leq 1$\egroup%
\lthtmlinlinemathZ
\lthtmlcheckvsize\clearpage}

{\newpage\clearpage
\lthtmlinlinemathA{tex2html_wrap_indisplay22145}%
$\displaystyle (\alpha c + (1 - \alpha) d) x.$%
\lthtmlindisplaymathZ
\lthtmlcheckvsize\clearpage}

{\newpage\clearpage
\lthtmlinlinemathA{tex2html_wrap_inline22147}%
\bgroup\color{Brown}$ \alpha$\egroup%
\lthtmlinlinemathZ
\lthtmlcheckvsize\clearpage}

{\newpage\clearpage
\lthtmlinlinemathA{tex2html_wrap_inline22149}%
\bgroup\color{Brown}$ x^c$\egroup%
\lthtmlinlinemathZ
\lthtmlcheckvsize\clearpage}

{\newpage\clearpage
\lthtmlinlinemathA{tex2html_wrap_inline22151}%
\bgroup\color{Brown}$ \alpha = 1$\egroup%
\lthtmlinlinemathZ
\lthtmlcheckvsize\clearpage}

{\newpage\clearpage
\lthtmlinlinemathA{tex2html_wrap_inline22153}%
\bgroup\color{Brown}$ x^d$\egroup%
\lthtmlinlinemathZ
\lthtmlcheckvsize\clearpage}

{\newpage\clearpage
\lthtmlinlinemathA{tex2html_wrap_inline22155}%
\bgroup\color{Brown}$ \alpha = 0$\egroup%
\lthtmlinlinemathZ
\lthtmlcheckvsize\clearpage}

{\newpage\clearpage
\lthtmlinlinemathA{tex2html_wrap_indisplay22159}%
$\displaystyle \max \{\alpha (cp - cx^c), (1 - \alpha)(dp - dx^d)\}.$%
\lthtmlindisplaymathZ
\lthtmlcheckvsize\clearpage}

\stepcounter{subsection}
{\newpage\clearpage
\lthtmlfigureA{figure3064}%
\begin{figure}\centering
\psfig{figure=lploop.eps,width=4.80in}

\end{figure}%
\lthtmlfigureZ
\lthtmlcheckvsize\clearpage}

\stepcounter{subsubsection}
\stepcounter{subsubsection}
\stepcounter{subsubsection}
\stepcounter{subsection}
\stepcounter{subsubsection}
\stepcounter{subsubsection}
\stepcounter{subsubsection}
\stepcounter{subsection}
\stepcounter{subsection}
\stepcounter{subsubsection}
\stepcounter{subsubsection}
\stepcounter{section}
\stepcounter{subsection}
\stepcounter{subsection}
\stepcounter{subsection}
\stepcounter{chapter}
\stepcounter{section}
\stepcounter{section}
\stepcounter{subsection}
\stepcounter{subsection}
\stepcounter{subsubsection}
\stepcounter{subsubsection}
\stepcounter{subsubsection}
\stepcounter{section}
\stepcounter{section}
\stepcounter{section}
\stepcounter{subsection}
\stepcounter{subsection}
\stepcounter{section}
\stepcounter{subsection}
\stepcounter{subsection}
\stepcounter{subsection}
\stepcounter{subsection}
\stepcounter{subsection}
\stepcounter{section}
{\newpage\clearpage
\lthtmlinlinemathA{tex2html_wrap_inline22227}%
\bgroup\color{Brown}$ (i,j)$\egroup%
\lthtmlinlinemathZ
\lthtmlcheckvsize\clearpage}

{\newpage\clearpage
\lthtmlinlinemathA{tex2html_wrap_inline22229}%
\bgroup\color{Brown}$ i<j$\egroup%
\lthtmlinlinemathZ
\lthtmlcheckvsize\clearpage}

\stepcounter{chapter}
\stepcounter{section}
\stepcounter{subsection}
\stepcounter{subsubsection}
\stepcounter{subsubsection}
\stepcounter{subsubsection}
\stepcounter{subsubsection}
\stepcounter{subsubsection}
\stepcounter{subsubsection}
{\newpage\clearpage
\lthtmlinlinemathA{tex2html_wrap_inline22315}%
$ \leq$%
\lthtmlinlinemathZ
\lthtmlcheckvsize\clearpage}

{\newpage\clearpage
\lthtmlinlinemathA{tex2html_wrap_inline22317}%
$ \geq$%
\lthtmlinlinemathZ
\lthtmlcheckvsize\clearpage}

\stepcounter{subsubsection}
\stepcounter{subsubsection}
\stepcounter{subsubsection}
\stepcounter{subsubsection}
\stepcounter{subsubsection}
\stepcounter{subsubsection}
\stepcounter{subsubsection}
\stepcounter{subsubsection}
\stepcounter{subsubsection}
\stepcounter{subsection}
\stepcounter{subsubsection}
\stepcounter{subsubsection}
\stepcounter{subsubsection}
\stepcounter{subsubsection}
\stepcounter{subsubsection}
\stepcounter{subsubsection}
\stepcounter{subsubsection}
\stepcounter{subsection}
\stepcounter{subsubsection}
\stepcounter{subsubsection}
\stepcounter{subsubsection}
\stepcounter{subsubsection}
\stepcounter{subsubsection}
\stepcounter{subsubsection}
\stepcounter{subsubsection}
\stepcounter{subsection}
\stepcounter{subsubsection}
\stepcounter{subsubsection}
\stepcounter{subsubsection}
\stepcounter{subsubsection}
\stepcounter{subsubsection}
\stepcounter{subsubsection}
\stepcounter{subsubsection}
\stepcounter{subsubsection}
\stepcounter{subsubsection}
\stepcounter{subsubsection}
\stepcounter{subsubsection}
\stepcounter{subsubsection}
\stepcounter{subsubsection}
\stepcounter{subsubsection}
\stepcounter{subsubsection}
\stepcounter{subsubsection}
\stepcounter{subsubsection}
\stepcounter{subsubsection}
\stepcounter{subsubsection}
\stepcounter{subsubsection}
\stepcounter{subsubsection}
\stepcounter{subsubsection}
\stepcounter{subsubsection}
\stepcounter{subsubsection}
\stepcounter{subsection}
\stepcounter{subsubsection}
\stepcounter{subsubsection}
\stepcounter{subsubsection}
\stepcounter{subsubsection}
\stepcounter{subsubsection}
\stepcounter{subsubsection}
\stepcounter{subsubsection}
\stepcounter{subsubsection}
\stepcounter{subsubsection}
\stepcounter{subsubsection}
\stepcounter{subsubsection}
\stepcounter{subsubsection}
\stepcounter{subsubsection}
\stepcounter{subsubsection}
\stepcounter{subsubsection}
\stepcounter{subsubsection}
\stepcounter{subsubsection}
\stepcounter{subsection}
\stepcounter{subsubsection}
\stepcounter{subsubsection}
\stepcounter{subsubsection}
\stepcounter{subsubsection}
\stepcounter{subsubsection}
\stepcounter{subsubsection}
\stepcounter{subsection}
\stepcounter{subsubsection}
\stepcounter{subsubsection}
\stepcounter{subsubsection}
\stepcounter{subsubsection}
\stepcounter{section}
\stepcounter{section}
\stepcounter{subsection}
\stepcounter{subsubsection}
\stepcounter{subsubsection}
\stepcounter{subsubsection}
\stepcounter{subsubsection}
\stepcounter{subsubsection}
\stepcounter{subsubsection}
\stepcounter{subsubsection}
\stepcounter{subsubsection}
\stepcounter{subsubsection}
\stepcounter{subsubsection}
\stepcounter{subsubsection}
\stepcounter{subsubsection}
\stepcounter{subsubsection}
\stepcounter{subsubsection}
\stepcounter{subsubsection}
\stepcounter{subsection}
\stepcounter{subsubsection}
\stepcounter{paragraph}
\stepcounter{subsubsection}
{\newpage\clearpage
\lthtmlinlinemathA{tex2html_wrap_inline24163}%
\bgroup\color{Brown}$ \times$\egroup%
\lthtmlinlinemathZ
\lthtmlcheckvsize\clearpage}

{\newpage\clearpage
\lthtmlinlinemathA{tex2html_wrap_inline24165}%
\bgroup\color{Brown}$ \leq$\egroup%
\lthtmlinlinemathZ
\lthtmlcheckvsize\clearpage}

{\newpage\clearpage
\lthtmlinlinemathA{tex2html_wrap_inline24167}%
\bgroup\color{Brown}$ =$\egroup%
\lthtmlinlinemathZ
\lthtmlcheckvsize\clearpage}

{\newpage\clearpage
\lthtmlinlinemathA{tex2html_wrap_inline24169}%
\bgroup\color{Brown}$ \geq$\egroup%
\lthtmlinlinemathZ
\lthtmlcheckvsize\clearpage}

{\newpage\clearpage
\lthtmlinlinemathA{tex2html_wrap_inline24171}%
\bgroup\color{Brown}$ 0^{th}$\egroup%
\lthtmlinlinemathZ
\lthtmlcheckvsize\clearpage}

{\newpage\clearpage
\lthtmlinlinemathA{tex2html_wrap_inline24173}%
\bgroup\color{Brown}$ 1^{st}$\egroup%
\lthtmlinlinemathZ
\lthtmlcheckvsize\clearpage}

\stepcounter{paragraph}
\stepcounter{subsubsection}
{\newpage\clearpage
\lthtmlinlinemathA{tex2html_wrap_inline24177}%
$ {\tt rhs}+{\tt range}$%
\lthtmlinlinemathZ
\lthtmlcheckvsize\clearpage}

{\newpage\clearpage
\lthtmlinlinemathA{tex2html_wrap_inline24181}%
$ =$%
\lthtmlinlinemathZ
\lthtmlcheckvsize\clearpage}

\stepcounter{paragraph}
\stepcounter{subsubsection}
\stepcounter{paragraph}
\stepcounter{subsubsection}
\stepcounter{subsubsection}
\stepcounter{subsubsection}
\stepcounter{subsubsection}
\stepcounter{subsubsection}
{\newpage\clearpage
\lthtmlinlinemathA{tex2html_wrap_inline24250}%
$ \epsilon$%
\lthtmlinlinemathZ
\lthtmlcheckvsize\clearpage}

\stepcounter{subsubsection}
\stepcounter{subsubsection}
\stepcounter{subsubsection}
\stepcounter{subsubsection}
\stepcounter{subsubsection}
\stepcounter{subsubsection}
\stepcounter{subsubsection}
\stepcounter{subsubsection}
\stepcounter{subsubsection}
\stepcounter{subsubsection}
\stepcounter{subsubsection}
\stepcounter{subsubsection}
\stepcounter{subsubsection}
{\newpage\clearpage
\lthtmlinlinemathA{tex2html_wrap_inline24766}%
$ -1$%
\lthtmlinlinemathZ
\lthtmlcheckvsize\clearpage}

\stepcounter{subsubsection}
\stepcounter{subsubsection}
\stepcounter{subsubsection}
{\newpage\clearpage
\lthtmlinlinemathA{tex2html_wrap_inline24893}%
$ ^{\tt th}$%
\lthtmlinlinemathZ
\lthtmlcheckvsize\clearpage}

\stepcounter{subsubsection}
\stepcounter{subsection}
\stepcounter{subsubsection}
\stepcounter{subsubsection}
\stepcounter{subsubsection}
\stepcounter{subsubsection}
\stepcounter{subsubsection}
\stepcounter{subsection}
\stepcounter{subsubsection}
\stepcounter{subsubsection}
\stepcounter{subsubsection}
\stepcounter{subsubsection}
\stepcounter{subsubsection}
\stepcounter{subsubsection}
\stepcounter{subsection}
\stepcounter{subsubsection}
\stepcounter{subsubsection}
\stepcounter{subsubsection}
\stepcounter{subsubsection}
\stepcounter{section}
\stepcounter{subsection}
\stepcounter{subsection}
\stepcounter{subsection}
\stepcounter{subsection}
\stepcounter{subsection}
\stepcounter{subsection}
\stepcounter{subsection}
\stepcounter{subsection}

\end{document}
