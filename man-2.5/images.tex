\batchmode
\documentclass[twoside,11pt]{article}
\makeatletter

\oddsidemargin 0in
\evensidemargin 0in
\topmargin 0in
\textwidth 6.5in
\newlength{\headwidth}
\setlength{\headwidth}{\textwidth}
\addtolength{\headwidth}{-4.4444pt}
\textheight 8.7in

\catcode`\@=11\relax   
\if@twoside         
\else               
\fi

\catcode`\@=12\relax   
\pagestyle{headings}

\markright{\bf \thesection}{}

\usepackage{html}
























\htmladdtonavigation{\htmladdnormallink 
	{\htmladdimg{contents_motif.gif}}{man.html}}


\def\ps@headings{\let\@mkboth\markboth
\def  \@oddfoot{}\def  \@evenfoot{}\def  \@evenhead{
\begin{tabular}{@{}c@{}}
\hspace*{\headwidth} \\ [-3ex]
\rm \thepage  \hfill  \sl \leftmark \\ 
\hline
\end{tabular}
} \def  \@oddhead{
\begin{tabular}{@{}c@{}}
\hspace*{\headwidth} \\ [-3ex]
\sl \rightmark \hfill \rm \thepage \\ 
\hline
\end{tabular}
} \def  \sectionmark##1{\markboth {\uppercase{\ifnum \c@secnumdepth >\z@
    \thesection\hskip 1em\relax \fi ##1}}{}}\def  \subsectionmark##1{\markright {\ifnum \c@secnumdepth >\@ne
          \thesubsection\hskip 1em\relax \fi ##1}}}

\def\ps@headings{\let\@mkboth\markboth
\def  \@oddfoot{}\def  \@evenfoot{}\def  \@oddhead{
\begin{tabular}{@{}c@{}}
\hspace*{\headwidth} \\ [-3ex]
\sl \rightmark \hfill \rm \thepage \\ 
\hline
\end{tabular}
} \def  \sectionmark##1{\markright {\uppercase{\ifnum \c@secnumdepth >\z@
    \thesection\hskip 1em\relax \fi ##1}}}}

\newcommand {\be}{\begin{enumerate}}

\newcommand {\ee}{\end{enumerate}}

\newcommand {\bc}{\begin{center}}

\newcommand {\ec}{\end{center}}

\newcommand {\bt}{\begin{tabular}}

\newcommand {\et}{\end{tabular}}

\newcommand {\bd}{\begin{description}}

\newcommand {\ed}{\end{description}}

\newcommand {\bi}{\begin{itemize}}

\newcommand {\ei}{\end{itemize}}

\newcommand {\bv}{\begin{verbatim}\end{verbatim}}

\newcommand {\functiondef}[1]{\subsubsection{#1}}

\newcommand {\describe}{\item[Description:] \hfill}

\newcommand {\args}{\item[Arguments:] \hfill}

\newcommand {\returns}{\item[Return values:] \hfill}

\newcommand {\postp}{\item[Post-processing:] \hfill}

\newcommand {\nopostp}{\item[Post-processing:] None \hfill}

\newcommand {\BB}{{\sc SYMPHONY}}

\newcommand {\TM}{{\sc TreeManager}}

\newcommand {\LP}{{\sc LP}}

\newcommand {\ra}{$\rightarrow$}

\makeatother
\newenvironment{tex2html_wrap}{}{}
\newwrite\lthtmlwrite
\def\lthtmltypeout#1{{\let\protect\string\immediate\write\lthtmlwrite{#1}}}%
\newbox\sizebox
\begin{document}
\pagestyle{empty}
\setcounter{page}{1}
{\newpage
\clearpage
\samepage % latex2html id marker 2099
\thesection
}

\stepcounter{section}
\stepcounter{subsection}
\stepcounter{subsection}
\stepcounter{subsection}
{\newpage
\clearpage
\samepage \begin{tabular}{lp{322.779pt}} 

{\tt ERROR} & Error in the user function. Printing an error message is
the user's responsibility. Depending on the work the user function was
supposed to do, the error might be ignored (and some default option
used), or the process aborts. \\ 

{\tt USER\_AND\_PP} & The user implemented both the user function and
post-processing (post-processing by {\sc SYMPHONY}\ will be skipped).\\ 

{\tt USER\_NO\_PP} & The user implemented the user function only. \\ 

{\tt DEFAULT} & The default option is going to be used (the default is one of
the built-in options, {\sc SYMPHONY}\ decides which one to use based on initial parameter
settings and the execution of the algorithm). \\ 

{\tt built\_in\_option1 } & \\ 
{\tt built\_in\_option2 } ... & The specified built-in option will be used.\\ 
\end{tabular}
}

\stepcounter{subsection}
\stepcounter{subsubsection}
\stepcounter{subsubsection}
\stepcounter{subsection}
\stepcounter{subsection}
\stepcounter{subsection}
{\newpage
\clearpage
\samepage \setbox\sizebox=\hbox{$^{\copyright}$}\lthtmltypeout{latex2htmlSize :tex2html_wrap_inline2100: \the\ht\sizebox::\the\dp\sizebox.}\box\sizebox
}

\stepcounter{subsection}
\stepcounter{subsection}
\stepcounter{subsection}
\stepcounter{section}
\stepcounter{subsection}
\stepcounter{subsubsection}
{\newpage
\clearpage
\samepage \begin{tabular}{llp{317.395pt}}
{\tt void **user} & OUT & Pointer to the user-defined data structure. \\ 
\end{tabular}
}

{\newpage
\clearpage
\samepage \begin{tabular}{lp{371.195pt}}
{\tt ERROR} & Error. {\sc SYMPHONY}\ stops. \\ 
{\tt USER\_NO\_PP} & Initialization is done. \\ 
\end{tabular}
}

\stepcounter{subsubsection}
{\newpage
\clearpage
\samepage \begin{tabular}{llp{317.395pt}}
{\tt void **user} & INOUT & Pointer to the user-defined data structure
(should be {\tt NULL} on return). \\ 
\end{tabular}
}

{\newpage
\clearpage
\samepage \begin{tabular}{lp{371.195pt}}
{\tt ERROR} & Ignored. This is probably not a fatal error.\\ 
{\tt USER\_NO\_PP} & Everything was freed successfully. \\ 
\end{tabular}
}

\stepcounter{subsubsection}
{\newpage
\clearpage
\samepage \begin{tabular}{llp{301.645pt}}
{\tt void *user} & IN & Pointer to the user-defined data structure. \\ 
{\tt char *filename} & IN & The name of the parameter file. \\ 
\end{tabular}
}

{\newpage
\clearpage
\samepage \begin{tabular}{lp{371.195pt}}
{\tt ERROR} & Error. {\sc SYMPHONY}\ stops. \\ 
{\tt USER\_NO\_PP} & User parameters were read successfully. \\ 
\end{tabular}
}

\stepcounter{subsubsection}
{\newpage
\clearpage
\samepage \begin{tabular}{llp{322.645pt}}
{\tt void *user} & IN & Pointer to the user-defined data structure. \\ 
\end{tabular}
}

{\newpage
\clearpage
\samepage \begin{tabular}{lp{371.195pt}}
{\tt ERROR} & Error. {\sc SYMPHONY}\ stops. \\ 
{\tt USER\_NO\_PP} & User I/O was completed successfully. \\ 
\end{tabular}
}

\stepcounter{subsubsection}
{\newpage
\clearpage
\samepage \begin{tabular}{llp{322.645pt}}
{\tt void *user} & IN & Pointer to the user-defined data structure. \\ 
{\tt int dg\_id} & IN & The process id of the graph drawing process. \\ 
\end{tabular}
}

{\newpage
\clearpage
\samepage \begin{tabular}{lp{371.195pt}}
{\tt ERROR} & Error. {\sc SYMPHONY}\ stops. \\ 
{\tt USER\_NO\_PP} & The user completed initialization successfully. \\ 
\end{tabular}
}

\stepcounter{subsubsection}
{\newpage
\clearpage
\samepage \begin{tabular}{llp{322.645pt}}
{\tt void *user} & IN & Pointer to the user-defined data structure. \\ 
{\tt int *ub} & OUT & Pointer to the global upper bound. Initially,
the upper bound is set to either {\tt -MAXDOUBLE} or the bound read in
from the parameter file, and should be changed by the user only if
a better valid upper bound is found.\\ 
\end{tabular}
}

{\newpage
\clearpage
\samepage \begin{tabular}{lp{371.195pt}}
{\tt ERROR} & Error. This error is probably not fatal. \\ 
{\tt USER\_NO\_PP} & User executed function successfully. \\ 
\end{tabular}
}

\stepcounter{subsubsection}
{\newpage
\clearpage
\samepage \begin{tabular}{llp{297.865pt}}
{\tt void *user} & IN & Pointer to the user-defined data structure. \\ 
{\tt int *varnum} & OUT & Pointer to the number of base variables. \\ 
{\tt int **userind} & OUT & Pointer to an array containing the user
indices of the base variables. \\ 
{\tt int **lb} & OUT & Pointer to an array containing the lower bounds for
the base variables. \\ 
{\tt int **ub} & OUT & Pointer to an array containing the upper bounds for
the base variables. \\ 
{\tt int *cutnum} & OUT & The number of base constraints. \\ 
{\tt int *colgen\_strat} & INOUT & The default strategy or one that has
been read in from the parameter file is passed in, but the user is free
to change it. See {\tt colgen\_strat} in the description of
parameters for details on how to set it.
\end{tabular}
}

{\newpage
\clearpage
\samepage \begin{tabular}{lp{371.195pt}}
{\tt ERROR} & Error. {\sc SYMPHONY}\ stops. \\ 
{\tt USER\_NO\_PP} & The required data are filled in, but no
post-processing done. \\ 
{\tt USER\_AND\_PP} & All required post-processing done. \\ 
\end{tabular}
}

\stepcounter{subsubsection}
{\newpage
\clearpage
\samepage \begin{tabular}{lp{371.195pt}}
{\tt ERROR} & Error. {\sc SYMPHONY}\ stops. \\ 
{\tt USER\_NO\_PP} & All required data  filled out, but no
post-processing done. \\ 
{\tt USER\_AND\_PP} & All required post-processing done. \\ 
\end{tabular}
}

{\newpage
\clearpage
\samepage \begin{tabular}{llp{297.865pt}}
{\tt void *user} & IN & Pointer to the user-defined data structure. \\ 
{\tt int *extravarnum} & OUT & Pointer to the number of extra active
variables in the root. \\ 
{\tt int *extravars} & OUT & Pointer to an array containing a list of
user indices of the extra variables to be active in the root. \\ 
\end{tabular}
}

\stepcounter{subsubsection}
{\newpage
\clearpage
\samepage \begin{tabular}{llp{301.645pt}}
{\tt void *user} & IN & Pointer to the user-defined data structure. \\ 
{\tt int msgtag} &    IN & {\tt FEASIBLE\_SOLUTION\_NONZEROS} or {\tt
FEASIBLE\_SOLUTION\_USER} \\ 
{\tt double cost}  &    IN & The cost of the feasible solution.\\ 
{\tt int numvars} &  IN & The number of variables whose user indices and
values were sent (length of {\tt indices} and {\tt values}). \\ 
{\tt int *indices} &  IN & The user indices of the nonzero variables. \\ 
{\tt double *values} & IN & The corresponding values. \\ 
\end{tabular}
}

{\newpage
\clearpage
\samepage \begin{tabular}{lp{371.195pt}}
{\tt ERROR} & Ignored. This is probably not a fatal error.\\ 
{\tt USER\_NO\_PP} & The solution has been unpacked and stored. \\ 
\end{tabular}
}

\stepcounter{subsubsection}
{\newpage
\clearpage
\samepage \begin{tabular}{llp{322.645pt}}
{\tt void *user} & IN & Pointer to the user-defined data structure. \\ 
{\tt void **user\_lp} & OUT & Pointer to the user-defined data
structure for the LP process. \\ 
\end{tabular}
}

{\newpage
\clearpage
\samepage \begin{tabular}{lp{371.195pt}}
{\tt ERROR} & Error. {\sc SYMPHONY}\ stops. \\ 
{\tt USER\_NO\_PP} & Packing is done. \\ 
\end{tabular}
}

\stepcounter{subsubsection}
{\newpage
\clearpage
\samepage \begin{tabular}{llp{322.645pt}}
{\tt void *user} & IN & Pointer to the user-defined data structure. \\ 
{\tt void **user\_cg} & OUT & Pointer to the user-defined data
structure for the cut generator process. \\ 
\end{tabular}
}

\stepcounter{subsubsection}
{\newpage
\clearpage
\samepage \begin{tabular}{llp{322.645pt}}
{\tt void *user} & IN & Pointer to the user-defined data structure. \\ 
{\tt void **user\_cp} & OUT & Pointer to the user-defined data
structure for the cut pool process. \\ 
\end{tabular}
}

\stepcounter{subsubsection}
{\newpage
\clearpage
\samepage \begin{tabular}{lp{371.195pt}}
{\tt ERROR} & Ignored. \\ 
{\tt USER\_NO\_PP} & Displaying is done. \\ 
\end{tabular}
}

\stepcounter{subsubsection}
{\newpage
\clearpage
\samepage \begin{tabular}{lp{371.195pt}}
{\tt ERROR} & Ignored. \\ 
{\tt USER\_NO\_PP} & Message is processed. \\ 
\end{tabular}
}

{\newpage
\clearpage
\samepage \begin{tabular}{llp{322.645pt}}
{\tt void *user} & IN & Pointer to the user-defined data structure. \\ 
{\tt int msgtag} & IN & The message tag of the message. \\ 
\end{tabular}
}

\stepcounter{subsection}
\stepcounter{subsubsection}
\stepcounter{paragraph}
{\newpage
\clearpage
\samepage \setbox\sizebox=\hbox{${\tt rhs}+{\tt range}$}\lthtmltypeout{latex2htmlSize :tex2html_wrap_inline2102: \the\ht\sizebox::\the\dp\sizebox.}\box\sizebox
}

{\newpage
\clearpage
\samepage \setbox\sizebox=\hbox{$\leq$}\lthtmltypeout{latex2htmlSize :tex2html_wrap_inline2104: \the\ht\sizebox::\the\dp\sizebox.}\box\sizebox
}

{\newpage
\clearpage
\samepage \setbox\sizebox=\hbox{$\geq$}\lthtmltypeout{latex2htmlSize :tex2html_wrap_inline2108: \the\ht\sizebox::\the\dp\sizebox.}\box\sizebox
}

\stepcounter{paragraph}
\stepcounter{paragraph}
\stepcounter{subsubsection}
{\newpage
\clearpage
\samepage \begin{tabular}{llp{317.395pt}}
{\tt void **user} & OUT & Pointer to the user-defined LP data
structure. \\ 
\end{tabular}
}

{\newpage
\clearpage
\samepage \begin{tabular}{lp{371.195pt}}
{\tt ERROR} & Error. {\sc SYMPHONY}\ aborts this LP process. \\ 
{\tt USER\_NO\_PP} & User received the data. \\ 
\end{tabular}
}

\stepcounter{subsubsection}
{\newpage
\clearpage
\samepage \begin{tabular}{llp{317.395pt}}
{\tt void **user} & INOUT & Pointer to the user-defined LP data structure. \\ 
\end{tabular}
}

{\newpage
\clearpage
\samepage \begin{tabular}{lp{371.195pt}}
{\tt ERROR} & Error. {\sc SYMPHONY}\ ignores error message. \\ 
{\tt USER\_NO\_PP} & User freed everything in the user space. \\ 
\end{tabular}
}

\stepcounter{subsubsection}
{\newpage
\clearpage
\samepage \setbox\sizebox=\hbox{$0^th$}\lthtmltypeout{latex2htmlSize :tex2html_wrap_inline2126: \the\ht\sizebox::\the\dp\sizebox.}\box\sizebox
}

{\newpage
\clearpage
\samepage \setbox\sizebox=\hbox{$1^st$}\lthtmltypeout{latex2htmlSize :tex2html_wrap_inline2128: \the\ht\sizebox::\the\dp\sizebox.}\box\sizebox
}

{\newpage
\clearpage
\samepage \begin{tabular}{llp{291.115pt}}
{\tt void *user} & IN & Pointer to the user-defined LP data structure. \\ 
& & \\ 

{\tt int varnum} & IN & Number of variables in the relaxation (base
and extra). \\ 

{\tt var\_desc **vars} & IN & An array of length {\tt n} containing the user
indices of the active variables (base and extra). \\ 

{\tt int rownum} & IN & Number of constraints in the relaxation (base
and extra). \\ 

{\tt int cutnum} & IN & Number of extra constraints. \\ 

{\tt cut\_data **cuts} & IN & Packed description of extra constraints. \\ 
& & \\ 

{\tt int *nz} & OUT &  Pointer to the number of nonzeros in the LP.\\ 

{\tt int **matbeg} & INOUT & Pointers to the arrays that describe the LP
problem (see description above. \\ 
{\tt int **matind} & INOUT & \\ 
{\tt double **matval} & INOUT & \\ 
{\tt double **obj} & INOUT & \\ 
{\tt double **rhs} & INOUT & \\ 
{\tt char **sense} & INOUT & \\ 
{\tt double **rngval} & INOUT & \\ 
& & \\ 
{\tt int *maxn} & INOUT & The maximum number of variables.\\ 
{\tt int *maxm} & INOUT & The maximum number of constraints.\\ 
{\tt int *maxnz} & INOUT & The maximum number of nonzeros. \\ 
& & \\ 
{\tt int *allocn} & INOUT & The length of the {\tt *matbeg} and {\tt *obj}
arrays (should be {\tt *maxm + *maxn +1}). \\ 
{\tt int *allocm} & INOUT & The length of the {\tt *rhs}, {\tt *sense}
and {\tt *rngval} arrays. \\ 
{\tt int *allocnz} & INOUT & The length of the {\tt *matval} and {\tt
*matind} arrays (should be {\tt *maxnz + *maxm}. \\ 
\end{tabular}
}

{\newpage
\clearpage
\samepage \begin{tabular}{lp{365.945pt}}
{\tt ERROR} & Error. The LP process is aborted. \\ 
{\tt USER\_AND\_PP} & Post-processing will be skipped, the user added the
constraints corresponding to the cuts. \\ 
{\tt USER\_NO\_PP} & User created the matrix with only the base constraints. \\ 
\end{tabular}
}

\stepcounter{subsubsection}
{\newpage
\clearpage
\samepage \begin{tabular}{llp{317.395pt}}
& & \\ 
{\tt void *user} & IN & Pointer to the user-defined LP data structure. \\ 

{\tt int varnum} & IN & Length of {\tt vars}. \\ 
{\tt int *vars} & IN & Array containing the user indices of the variables. \\ 
{\tt double *ub} & OUT & Array of upper bounds (to be filled out by
the user).
\\ 
\end{tabular}
}

{\newpage
\clearpage
\samepage \begin{tabular}{lp{371.195pt}}
{\tt ERROR} & Error. The LP process is aborted. \\  
{\tt DEFAULT} & Upper bounds are set to one. \\  
{\tt USER\_NO\_PP} & The user filled up the upper bound array. \\ 
\end{tabular}
}

\stepcounter{subsubsection}
{\newpage
\clearpage
\samepage \begin{tabular}{llp{301.645pt}}
{\tt void *user} & INOUT & Pointer to the user-defined LP data structure. \\ 
& & \\ 
{\tt double lpetol} & IN & The $\epsilon$ tolerance of the LP solver. \\ 
{\tt int varnum} & IN & The length of the {\tt indices} and {\tt values}
arrays.\\  
{\tt int *indices} & IN & User indices of variables at nonzero level in the
current solution.\\  
{\tt double *values} & IN & Values of the variables listed in {\tt indices}.\\ 
& & \\ 
{\tt int *feasible} & OUT & Feasibility status of the solution ({\tt
NOT\_FEASIBLE}, or {\tt FEASIBLE}). \\ 
\end{tabular}
}

{\newpage
\clearpage
\samepage \begin{tabular}{lp{338.226pt}}
{\tt ERROR} & Error. Solution is considered to be not feasible.\\ 
{\tt USER\_NO\_PP} & User checked IP feasibility. \\ 
{\tt DEFAULT} & Regulated by the parameter {\tt is\_feasible\_default},
but set to {\tt TEST\_INTEGRALITY} unless over-ridden by the user.\\ 
{\tt TEST\_INTEGRALITY} & Test integrality of the given solution. \\ 
{\tt TEST\_ZERO\_ONE} & Tests whether the solution is binary. \\ 
\end{tabular}
}

\stepcounter{subsubsection}
{\newpage
\clearpage
\samepage \begin{tabular}{llp{301.645pt}}
{\tt void *user} & IN & Pointer to the user-defined LP data structure. \\ 
& & \\ 
{\tt double lpetol} & IN & The $\epsilon$ tolerance of the LP solver. \\ 
{\tt int varnum} & IN & The length of the {\tt indices} and {\tt
values} arrays.\\ 
{\tt int *indices} & IN & User indices of variables at nonzero level
in the current solution.\\ 
{\tt double *values} & IN & Values of the variables listed in {\tt indices}.\\ 
\end{tabular}
}

{\newpage
\clearpage
\samepage \begin{tabular}{lp{353.976pt}}
{\tt ERROR} & Error. Do the default.\\ 
{\tt USER\_NO\_PP} & User packed the solution.\\ 
{\tt DEFAULT} & Regulated by the parameter {\tt
pack\_feasible\_solution\_default}, but set to {\tt SEND\_NONZEROS}
unless over-ridden by the user.\\ 
{\tt SEND\_NONZEROS} & Pack the nonzero values and their indices.\\ 
\end{tabular}
}

\stepcounter{subsubsection}
{\newpage
\clearpage
\samepage \begin{tabular}{llp{301.645pt}}
{\tt void *user} & IN & Pointer to the user-defined LP data structure.\\ 
& & \\ 
{\tt int which\_sol} & IN & The type of solution passed on to the
displaying function. Possible values are {\tt DISP\_FEAS\_SOLUTION},
{\tt DISP\_RELAXED\_SOLUTION} and {\tt DISP\_FINAL\_RELAXED\_SOLUTION}. \\ 
{\tt int varnum} & IN & The number of variables in the current
solution at nonzero level (the length of the {\tt indices} and {\tt
values} arrays). \\ 
{\tt int *indices} & IN & User indices of variables at
nonzero level in the current solution.\\ 
{\tt double *values} & IN & Values of the nonzero variables.\\ 
\end{tabular}
}

{\newpage
\clearpage
\samepage \begin{tabular}{lp{350.196pt}}
{\tt ERROR} & Error. {\sc SYMPHONY}\ ignores error message. \\ 
{\tt USER\_NO\_PP} & User displayed whatever she wanted to.\\ 
{\tt DEFAULT} & Regulated by the parameter {\tt display\_solution\_default}. \\ 
{\tt DISP\_NOTHING} & Display nothing. \\ 
{\tt DISP\_NZ\_INT} & Display user indices (as integers) and values of
nonzero variables. \\ 
{\tt DISP\_NZ\_HEXA} & Display user indices (as hexadecimals) and
values of nonzero variables. \\ 
{\tt DISP\_FRAC\_INT} & Display user indices (as integers) and values
of variables not at their lower or upper bounds. \\ 
{\tt DISP\_FRAC\_HEXA} & Display user indices (as hexadecimals) and
values of variables not at their lower and upper bounds. \\ 
\end{tabular}
}

\stepcounter{subsubsection}
{\newpage
\clearpage
\samepage \begin{tabular}% latex2html id marker 1374
{llp{237.805pt}}
{\tt void *user} &  IN & Pointer to the user-defined LP data structure. \\ 
{\tt double lpetol} & IN & The $\epsilon$ tolerance of the LP solver. \\ 
& & \\ 
{\tt int cutnum} & IN & The number of violated cuts (known before invoking
this function) that could be added to the problem (instead of branching).\\ 
& & \\ 
{\tt int slacks\_in\_matrix\_num} & IN & Number of slack constraints in the
matrix. \\ 
{\tt cut\_data **slacks\_in\_matrix} & IN & The description of the cuts
corresponding to these constraints (see Section \ref{cut-data-description}). \\ 
& & \\ 
{\tt int slack\_cut\_num} & IN & The number of slack cuts not in the
matrix. \\ 
{\tt cut\_data **slack\_cuts} & IN & Array of pointers to these cuts
(see Section \ref{cut-data-description}).\\ 
\end{tabular}
}

{\newpage
\clearpage
\samepage \begin{tabular}{llp{260.256pt}}
{\tt int varnum} & IN & The number of variables in the current lp
relaxation (the length of the following three arrays).\\ 
{\tt var\_desc **vars} & IN & Description of the variables in the
relaxation. \\ 
{\tt double *x} & IN & The corresponding solution values (in the optimal
solution to the relaxation).\\ 
{\tt char *status} & IN & The stati of the variables. 
There are five possible status values: 
{\tt NOT\_\-FIXED}, {\tt TEMP\_\-FIXED\_\-TO\_\-UB}, {\tt
PERM\_\-FIXED\_\-TO\_\-UB}, {\tt TEMP\_\-FIXED\_\-TO\_\-LB} and {\tt
PERM\_\-FIXED\_\-TO\_\-LB}.\\  
& & \\ 
{\tt int *cand\_num} & OUT & Pointer to the number of candidates
returned (the length of {\tt *candidates}).\\ 
{\tt candidate ***candidates} & OUT & Pointer to the array of
candidates generated (see description below).\\ 
{\tt int *action} & OUT & What to do. Must be one of the four above described
values.
\end{tabular}
}

{\newpage
\clearpage
\samepage \begin{tabular}{lp{371.195pt}}
{\tt ERROR} & Error. {\tt DEFAULT} is used. \\ 
{\tt USER\_NO\_PP} & The user filled out {\tt *action} (and possibly {\tt
*cand\_num} and {\tt *candidates}). \\ 
{\tt DEFAULT} & action is set to the value of the parameter {\tt
shall\_we\_branch\_default}, which is initially
{\tt USER\_\_BRANCH\_IF\_MUST} unless over-ridden by the user. \\  
\end{tabular}
}

\stepcounter{subsubsection}
{\newpage
\clearpage
\samepage \begin{tabular}{lp{256.327pt}}
{\tt ERROR} & Error. {\tt DEFAULT} is used. \\ 
{\tt USER\_NO\_PP} & User generated branching candidates. \\ 
{\tt DEFAULT} & Regulated by the {\tt select\_candidates\_default} parameter
(one of the following three options).\\ 
{\tt USER\_\_CLOSE\_TO\_HALF} & Choose variables with values closest to half.\\ 
{\tt USER\_\_CLOSE\_TO\_HALF\_AND\_EXPENSIVE} & Choose variables
with values close to half and with high objective function coefficients.\\ 
{\tt USER\_\_CLOSE\_TO\_ONE\_AND\_CHEAP} & Choose variables with
values close to one and with low objective function coefficients.\\ 
\end{tabular}
}

\stepcounter{subsubsection}
{\newpage
\clearpage
\samepage \begin{tabular}{llp{273.085pt}}
{\tt void *user} &  IN & Pointer to the user-defined LP data structure. \\ 
& & \\ 
{\tt branch\_obj *can1} & IN & One of the candidates to be compared.\\ 
{\tt branch\_obj *can2} & IN & The other candidate to be compared. \\ 
{\tt int *which\_is\_better} & OUT & The user's choice. See the description
above. \\ 
\end{tabular}
}

{\newpage
\clearpage
\samepage \begin{tabular}{lp{327.725pt}}
{\tt ERROR} & Error. {\tt DEFAULT} is used. \\ 
{\tt USER\_NO\_PP} & User filled out {\tt *which\_is\_better}. \\ 
{\tt DEFAULT} & Regulated by the {\tt compare\_candidates\_default}
parameter, initially set to {\tt LOWEST\_LOW\_OBJ} unless over-ridden by the
user. \\ 
{\tt BIGGEST\_DIFFERENCE} & Prefer the candidate with the biggest
difference between highest and lowest objective function values.\\ 
{\tt LOWEST\_LOW} & Prefer the candidate with the lowest minimum
objective function value. The minimum is taken over the objective function
values of all the children. \\ 
{\tt HIGHEST\_LOW} & Prefer the candidate with the highest minimum
objective function value. \\ 
{\tt LOWEST\_HIGH} & Prefer the candidate with the lowest maximum
objective function value. \\ 
{\tt HIGHEST\_HIGH} & Prefer the candidate with the highest maximum
objective function value .\\ 
\end{tabular}
}

\stepcounter{subsubsection}
{\newpage
\clearpage
\samepage \begin{tabular}{llp{297.864pt}}
{\tt void *user} &  IN & Pointer to the user-defined LP data structure. \\ 
& & \\ 
{\tt int ub} & IN & The current best upper bound. \\ 
{\tt double etol} & IN & Epsilon tolerance. \\ 
{\tt branch\_obj *can} & IN & The branching candidate. \\ 
& & \\ 
{\tt char *action} & OUT & Array of actions for the children. The array is
already allocated to length {\tt can->number}.\\ 
\end{tabular}
}

{\newpage
\clearpage
\samepage \begin{tabular}{lp{304.416pt}}
{\tt ERROR} & Error. {\tt DEFAULT} is used. \\ 
{\tt USER\_NO\_PP} & User filled out {\tt *action}. \\ 
{\tt USER\_AND\_PP} & User filled out {\tt *action} and did an
equivalent of the post-processing.\\ 
{\tt DEFAULT} & Regulated by the {\tt select\_child\_default}
parameter, which is initially set to {\tt PREFER\_LOWER\_OBJ\_VALUE},
unless over-ridden by the user. \\ 
{\tt PREFER\_HIGHER\_OBJ\_VALUE} & Choose child with the highest
objective value.\\ 
{\tt PREFER\_LOWER\_OBJ\_VALUE} & Choose child with the lowest
objective value.\\ 
\end{tabular}
}

\stepcounter{subsubsection}
{\newpage
\clearpage
\samepage \begin{tabular}{llp{297.865pt}}
{\tt void *user} &  IN & Pointer to the user-defined LP data structure. \\ 
& & \\ 
{\tt branch\_obj *can} & IN & The branching candidate. \\ 
{\tt cut\_data *cut} & IN & The description of the cut if the
branching object is a cut. \\ 
{\tt char *action} & IN & Array of actions for the children. \\ 
\end{tabular}
}

{\newpage
\clearpage
\samepage \begin{tabular}{lp{371.195pt}}
{\tt ERROR} & Error. Ignored by {\sc SYMPHONY}. \\ 
{\tt USER\_NO\_PP} & The user printed out whatever she wanted to. \\ 
\end{tabular}
}

\stepcounter{subsubsection}
{\newpage
\clearpage
\samepage \begin{tabular}{llp{297.864pt}}
{\tt void *user} & IN & Pointer to the user-defined LP data structure. \\ 
& & \\ 
{\tt int *desc\_size} & OUT & The size of the additional information, the
length of {\tt *desc} in bytes. \\ 
{\tt char **desc} & OUT & Pointer to the additional information (space must be
allocated by the user).\\ 
\end{tabular}
}

{\newpage
\clearpage
\samepage \begin{tabular}{lp{371.195pt}}
{\tt ERROR} & Error. {\tt DEFAULT} is used. \\ 
{\tt USER\_NO\_PP} & User filled out {\tt *desc\_size} and {\tt *desc}.\\ 
{\tt DEFAULT} & No description is appended. \\ 
\end{tabular}
}

\stepcounter{subsubsection}
{\newpage
\clearpage
\samepage \begin{tabular}{llp{303.114pt}}
{\tt void *user} & IN & Pointer to the user-defined LP data structure.\\ 
& & \\ 
{\tt cut\_data *cut1} & IN & The first cut.\\ 
{\tt cut\_data *cut2} & IN & The second cut.\\ 
{\tt int *same\_cuts} & OUT & Possible values: {\tt SAME}, {\tt
FIRST\_CUT\_BETTER}, {\tt SECOND\_CUT\_BETTER} and {\tt DIFFERENT} (i.e., not
comparable).\\ 
\end{tabular}
}

{\newpage
\clearpage
\samepage \begin{tabular}{lp{295.386pt}}
{\tt ERROR} & Error. {\tt DEFAULT} is used.\\ 
{\tt USER\_NO\_PP} & User did the comparison, filled out {\tt *same\_cuts}. \\ 
{\tt DEFAULT} & Compare byte by byte (see above). \\ 
\end{tabular}
}

\stepcounter{subsubsection}
{\newpage
\clearpage
\samepage \begin{tabular}{llp{252.085pt}}
{\tt void *user} & IN & Pointer to the user-defined LP data structure. \\ 
& & \\ 
{\tt int from} & IN & See below in ``Notes''. \\ 
{\tt int one\_row\_only} & IN & {\tt UNPACK\_CUTS\_SINGLE} or {\tt
UNPACK\_CUTS\_MULTIPLE} (see notes below). \\ 
{\tt int varnum} & IN & The number of variables. \\ 
{\tt var\_desc **vars} & IN & The variables currently in the problem. \\ 
{\tt int cutnum} & IN & The number of cuts to be decoded. \\ 
{\tt cut\_data **cuts} & IN & Cuts that need to be converted to rows
for the current LP. See ``Warning'' below. \\ 
& & \\ 
{\tt int *new\_row\_num} & OUT & Pointer to the number of rows in
{\tt **new\_rows}. \\ 
{\tt waiting\_row ***new\_rows} & OUT & Pointer to the array of
pointers to the new rows. \\ 
\end{tabular}
}

{\newpage
\clearpage
\samepage \begin{tabular}{lp{371.195pt}}
{\tt ERROR} & Error. The cuts are discarded.\\ 
{\tt USER\_NO\_PP} & User unpacked the cuts.\\ 
\end{tabular}
}

\stepcounter{subsubsection}
{\newpage
\clearpage
\samepage \begin{tabular}{llp{297.864pt}}
{\tt void *user} &  IN & Pointer to the user-defined LP data structure. \\ 
& & \\ 
{\tt int varnum} & IN & The number of variables currently in the LP
relaxation. (The length of the {\tt *vars} and {\tt x} arrays.) \\ 
{\tt var\_desc **vars} & IN & The variables currently in the LP relaxation.\\ 
{\tt double *x} & IN & Values of the above variables.\\ 
{\tt int where} & IN & Where the solution is to be sent---{\tt LP\_SOL\_TO\_CG} or {\tt LP\_SOL\_TO\_CP}. \\ 
\end{tabular}
}

{\newpage
\clearpage
\samepage \begin{tabular}{lp{348.725pt}}
{\tt ERROR} & Error. No message will be sent. \\ 
{\tt USER\_NO\_PP} & User packed and sent the message. \\ 
{\tt DEFAULT} & Regulated by the {\tt pack\_lp\_solution\_default}
parameter, initially set to {\tt SEND\_NOZEROS}. \\ 
{\tt SEND\_NONZEROS} & Send user indices and values of variables
at nonzero level.\\ 
{\tt SEND\_FRACTIONS} & Send user indices and values of variables
at fractional level.\\ 
\end{tabular}
}

\stepcounter{subsubsection}
{\newpage
\clearpage
\samepage \begin{tabular}{llp{297.864pt}}
{\tt void *user} &  IN & Pointer to the user-defined LP data structure. \\ 
& & \\ 
{\tt int varnum} & IN & The number of variables currently in the LP
relaxation. (The length of the {\tt *vars} and {\tt x} arrays.) \\ 
{\tt var\_desc **vars} & IN & The variables currently in the LP relaxation.\\ 
{\tt double *x} & IN & Values of the above variables.\\ 
{\tt char *status} & INOUT & Stati of variables currently in the LP
relaxation. \\ 
\end{tabular}
}

{\newpage
\clearpage
\samepage \begin{tabular}{lp{371.195pt}}
{\tt ERROR} & Error. Ignored by {\sc SYMPHONY}. \\ 
{\tt USER\_NO\_PP} & User changed the stati of the variables she
wanted. \\ 
\end{tabular}
}

\stepcounter{subsubsection}
{\newpage
\clearpage
\samepage \begin{tabular}{llp{287.365pt}}
{\tt void *user} &  IN & Pointer to the user-defined LP data structure. \\ 
& & \\ 
{\tt int generate\_what} & IN & {\tt GENERATE\_NEXTIND} or {\tt
GENERATE\_REAL\_NEXTIND} (see description above). \\ 
{\tt int cutnum} & IN & The number of added rows in the LP formulation
(i.e., the total number of rows less the number of base constraints).
This is the length of the {\tt **cuts} array.\\  
{\tt cut\_data **cuts} & IN & Description of the cuts corresponding to
the added rows of the current LP formulation. The user is supposed to
know about the cuts corresponding to the base constraints. \\ 
{\tt int prevind} & IN & The last variable processed ($-1$ if there was
none) by {\sc SYMPHONY}.\\ 
{\tt int nextind} & IN & The next variable ($-1$ if there are none)
known to {\sc SYMPHONY}.\\ 
& & \\ 
{\tt int *real\_nextind} & OUT & Pointer to the user index of the next
variable ($-1$ if there is none). \\ 
{\tt double *colval} & OUT & Values of the nonzero entries in the
column of the next variable. (Sufficient space is already allocated for this
array.) \\ 
{\tt int *colind} & OUT & Row indices of the nonzero entries in the column.
(Sufficient space is already allocated for this array.) \\ 
{\tt int *collen} & OUT & The length of the {\tt colval} and {\tt
colind} arrays.\\ 
{\tt double *obj} & OUT & Objective coefficient corresponding to the
next variable.\\ 
\end{tabular}
}

{\newpage
\clearpage
\samepage \begin{tabular}{lp{371.195pt}}
{\tt ERROR} & Error. The LP process is aborted.\\ 
{\tt USER\_NO\_PP} & User filled out {\tt *real\_nextind} and
generated its column if needed.\\ 
\end{tabular}
}

\stepcounter{subsubsection}
{\newpage
\clearpage
\samepage \begin{tabular}{llp{257.334pt}}
{\tt void *user} &  IN & Pointer to the user-defined LP data structure. \\ 
& & \\ 
{\tt int varnum} & IN & The number of variables currently in the LP
relaxation. (The length of the {\tt *vars} and {\tt x} arrays.) \\ 
{\tt var\_desc **vars} & IN & The variables currently in the LP relaxation.\\ 
{\tt double *x} & IN & Values of the above variables.\\ 
{\tt int *new\_row\_num} & OUT & The number of cuts generated. \\ 
{\tt waiting\_row ***new\_rows} & OUT & The cuts and the corresponding rows. \\ 
\end{tabular}
}

{\newpage
\clearpage
\samepage \begin{tabular}{lp{365.945pt}}
{\tt ERROR} & Error. Interpreted as if no cuts were generated. \\ 
{\tt USER\_NO\_PP} & Cuts were generated but {\sc SYMPHONY}\ must compare them to those
in the local pool. \\ 
{\tt USER\_AND\_PP} & Cuts were generated and {\sc SYMPHONY}\ should not compare them to
those in the local pool. \\ 
{\tt DEFAULT} & No cuts are generated. (At least for now. We might add Gomory
cuts for default later.) \\ 
\end{tabular}
}

\stepcounter{subsubsection}
{\newpage
\clearpage
\samepage \begin{tabular}{llp{282.115pt}}
{\tt void *user} &  IN & Pointer to the user-defined LP data structure. \\ 
& & \\ 
{\tt int rownum} & IN & The number of cuts added. \\ 
{\tt waiting\_row **rows} & IN & Array of waiting rows containing the
cuts added. \\ 
\end{tabular}
}

{\newpage
\clearpage
\samepage \begin{tabular}{lp{365.945pt}}
{\tt ERROR} & Revert to default. \\ 
{\tt USER\_AND\_PP} & User printed whatever he wanted. \\ 
{\tt DEFAULT} & Print out the number of cuts added. \\ 
\end{tabular}
}

\stepcounter{subsubsection}
{\newpage
\clearpage
\samepage \setbox\sizebox=\hbox{$^{\tt th}$}\lthtmltypeout{latex2htmlSize :tex2html_wrap_inline2150: \the\ht\sizebox::\the\dp\sizebox.}\box\sizebox
}

{\newpage
\clearpage
\samepage \begin{tabular}{llp{282.115pt}}
{\tt void *user} &  IN & Pointer to the user-defined LP data structure. \\ 
& & \\ 
{\tt int rownum} & IN & The number of waiting rows. \\ 
{\tt waiting\_row **rows} & IN & The array of waiting rows.\\ 
{\tt char *delete} & OUT & An array of indicators (each of them is one {\tt
char}) showing which waiting rows are to be deleted. \\  
\end{tabular}
}

{\newpage
\clearpage
\samepage \begin{tabular}{lp{365.945pt}}
{\tt ERROR} & Purge every single waiting row. \\ 
{\tt USER\_AND\_PP} & The user removed the unwanted waiting rows and
compressed the remaining list. \\ 
{\tt USER\_NO\_PP} & The user marked in {\tt delete} the rows to be deleted. \\ 
{\tt DEFAULT} & Described above. \\ 
\end{tabular}
}

\stepcounter{subsection}
\stepcounter{subsubsection}
{\newpage
\clearpage
\samepage \begin{tabular}{llp{317.395pt}}
{\tt void **user} & INOUT & Pointer to the user-defined data structure. \\ 
\end{tabular}
}

{\newpage
\clearpage
\samepage \begin{tabular}{lp{365.945pt}}
{\tt ERROR} & Error. CG exits. \\ 
{\tt USER\_NO\_PP} & The user received the data properly. \\ 
\end{tabular}
}

\stepcounter{subsubsection}
{\newpage
\clearpage
\samepage \begin{tabular}{llp{317.395pt}}
{\tt void *user} & IN & Pointer to the user-defined data structure. \\ 
\end{tabular}
}

{\newpage
\clearpage
\samepage \begin{tabular}{lp{365.945pt}}
{\tt ERROR} & Error. This LP solution is not processed. \\ 
{\tt USER\_NO\_PP} & The user received the LP solution. \\ 
\end{tabular}
}

\stepcounter{subsubsection}
{\newpage
\clearpage
\samepage \begin{tabular}{llp{317.395pt}}
{\tt void **user} & INOUT & Pointer to the user-defined data structure
(should be {\tt NULL} on exit from this function). \\ 
\end{tabular}
}

{\newpage
\clearpage
\samepage \begin{tabular}{lp{365.945pt}}
{\tt ERROR} & Ignored. \\ 
{\tt USER\_NO\_PP} & The user freed all data structures. \\ 
\end{tabular}
}

\stepcounter{subsubsection}
{\newpage
\clearpage
\samepage \begin{tabular}{llp{306.895pt}}
{\tt void *user} & IN & Pointer to the user-defined data structure.
\\ 
{\tt int iter\_num} & IN & The iteration number of the current LP solution. \\ 
{\tt int level} & IN & The level in the tree on which the current LP
solution was generated. \\ 
{\tt index} & IN & The index of the node in which LP solution was generated.
\\ 
{\tt objval} & IN & The objective function value of the current LP solution.
\\ 
{\tt int varnum} & IN & The number of nonzeros in the current LP solution. \\ 
{\tt indices} & IN & The column indices of the nonzero variables in the current
LP solution. \\ 
{\tt values} & IN & The values of the nonzero variables listed in 
{\tt indices}.
\\ 
{\tt double ub} & IN & The current global upper bound. \\ 
{\tt double lpetol} & IN & The current error tolerance in the LP. \\ 
{\tt int *cutnum} & OUT & Pointer to the number of cuts generated
and sent to the LP. \\ 
\end{tabular}
}

{\newpage
\clearpage
\samepage \begin{tabular}{lp{365.945pt}}
{\tt ERROR} & Ignored. \\ 
{\tt USER\_NO\_PP} & The user function exited properly. \\ 
\end{tabular}
}

\stepcounter{subsection}
\stepcounter{subsubsection}
{\newpage
\clearpage
\samepage \begin{tabular}{llp{313.615pt}}
{\tt void **user} & INOUT & Pointer to the user-defined data structure. \\ 
\end{tabular}
}

{\newpage
\clearpage
\samepage \begin{tabular}{lp{365.945pt}}
{\tt ERROR} & Error. Cut Pool exits. \\ 
{\tt USER\_NO\_PP} & The user received data successfully. \\ 
\end{tabular}
}

\stepcounter{subsubsection}
{\newpage
\clearpage
\samepage \begin{tabular}{llp{313.615pt}}
{\tt void **user} & INOUT & Pointer to the user-defined data structure
(should be {\tt NULL} on exit). \\ 
\end{tabular}
}

\stepcounter{subsubsection}
{\newpage
\clearpage
\samepage \begin{tabular}{llp{313.635pt}}
{\tt void *user} & IN & Pointer to the user-defined data structure. \\ 
\end{tabular}
}

{\newpage
\clearpage
\samepage \begin{tabular}{lp{365.945pt}}
{\tt ERROR} & Cuts are not checked for this LP solution. \\ 
{\tt USER\_NO\_PP} & The user function exited properly. \\ 
\end{tabular}
}

\stepcounter{subsubsection}
{\newpage
\clearpage
\samepage \begin{tabular}{llp{301.645pt}}
{\tt void *user} & IN & Pointer to the user-defined data structure. \\ 
{\tt int varnum} & IN & The number of nonzero/fractional variables described
in {\tt indices} and {\tt values}. \\ 
{\tt int *indices} & IN & The user indices of the nonzero/fractional
variables. \\ 
{\tt double *values} & IN & The nonzero/fractional values. \\  
\end{tabular}
}

{\newpage
\clearpage
\samepage \begin{tabular}{lp{365.945pt}}
{\tt ERROR} & Cuts are not checked for this LP solution. \\ 
{\tt USER\_NO\_PP} & The user is prepared to check cuts. \\ 
\end{tabular}
}

\stepcounter{subsubsection}
{\newpage
\clearpage
\samepage \begin{tabular}{llp{292.615pt}}
{\tt void *user} & INOUT & The user defined part of p. \\ 
{\tt double lpetol} & IN & The $\epsilon$ tolerance in the LP process. \\ 
{\tt int varnum} & IN & Same as the previous function. \\ 
{\tt int *indices} & IN & Same as the previous function. \\ 
{\tt double *values} & IN & Same as the previous function. \\ 
{\tt cut\_data *cut} & IN & Pointer to the cut to be tested. \\ 
{\tt int *is\_violated} & OUT & TRUE/FALSE based on whether the cut is violated
or not. \\ 
\end{tabular}
}

{\newpage
\clearpage
\samepage \begin{tabular}{lp{365.945pt}}
{\tt ERROR} & Cut is not sent to the LP, regardless of the value of
{\tt *is\_violated}. \\ 
{\tt USER\_NO\_PP} & The user function exited properly. \\ 
\end{tabular}
}

\stepcounter{subsubsection}
\stepcounter{section}
\stepcounter{subsection}
\stepcounter{subsection}
\stepcounter{subsection}
\stepcounter{subsection}
\stepcounter{subsection}
\stepcounter{subsection}
\stepcounter{section}

\end{document}
