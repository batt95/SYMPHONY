\batchmode


\documentclass[twoside,11pt]{book}
\RequirePackage{ifthen}




\usepackage{html}


\oddsidemargin 0in
\evensidemargin 0in
\topmargin 0in
\textwidth 6.5in

\newlength{\headwidth} 

\setlength{\headwidth}{\textwidth} 

\addtolength{\headwidth}{-4.4444pt} 
\textheight 8.7in
\setcounter{secnumdepth}{3}


\catcode`\@=11\relax   %allow @ in macro names


\if@twoside         % If two-sided printing.
\else               % If one-sided printing.

\fi


\catcode`\@=12\relax   %disable @ in macro names


\pagestyle{headings}


\markright{\bf\thesection }{}


\usepackage{graphicx}
\usepackage{amsmath}
\usepackage{amssymb}
\usepackage{fullpage}
\usepackage{easyeqn}
\usepackage{fancyvrb}
\usepackage{listings}
\usepackage{url}
\usepackage[latin1]{inputenc}
\usepackage[dvipsnames]{xcolor}
\usepackage{splitidx}
\makeindex
\newindex[Index of C API Functions]{f}
\newindex[Index of User Callback API Functions]{cf}
\newindex[Index of Parameters]{p}%
\providecommand{\mysindex}[2]{\sindex[#1]{#2}} 
\usepackage{xcolor}%
\providecommand{\mysindex}[2]{} 

\lstloadlanguages{C++}

%
\providecommand{\be}{\begin{enumerate}}%
\providecommand{\ee}{\end{enumerate}}%
\providecommand{\bc}{\begin{center}}%
\providecommand{\ec}{\end{center}}%
\providecommand{\bt}{\begin{tabular}}%
\providecommand{\et}{\end{tabular}}%
\providecommand{\bd}{\begin{description}}%
\providecommand{\ed}{\end{description}}%
\providecommand{\bi}{\begin{itemize}}%
\providecommand{\ei}{\end{itemize}}%
\providecommand{\bv}{\begin{verbatim}
}
\newcommand{\ev}{\end{verbatim}
}%
\providecommand{\functiondef}[1]{\subsubsection{#1}}%
\providecommand{\firstfuncdef}[1]{\subsubsection{#1}}%
\providecommand{\code}[1]{{\color{brown}\texttt{#1}}}%
\providecommand{\describe}{\item[{\color{cyan}Description:}] \hfill}%
\providecommand{\args}{\item[{\color{brown}Arguments:}] \hfill}%
\providecommand{\returns}{\item[{\color{green}Return values:}] \hfill}%
\providecommand{\postp}{\item[Post-processing:] \hfill}%
\providecommand{\nopostp}{\item[Post-processing:] None \hfill} 

%
\providecommand{\BB}{{\sc SYMPHONY}}%
\providecommand{\TM}{{\sc TreeManager}}%
\providecommand{\LP}{{\sc LP}}%
\providecommand{\ra}{$\rightarrow$}%
\providecommand{\vmin}{\mathop{\text{vmin}}}%
\providecommand{\Z}{\mathbb{Z}}%
\providecommand{\Q}{\mathbb{Q}} 
%
\renewcommand{\Re}{\mathbb{R}}%
\providecommand{\ptt}[1]{{\tt {\color{BrickRed} #1}}}%
\providecommand{\ptt}[1]{{\tt {\color{red} #1}}}%
\providecommand{\bs}{{$\backslash$}} 



\setlength{\parindent}{0in} 

\setlength{\parskip}{0.1in} 






\makeatletter

\makeatletter
\count@=\the\catcode`\_ \catcode`\_=8 
\newenvironment{tex2html_wrap}{}{}%
\catcode`\<=12\catcode`\_=\count@
\newcommand{\providedcommand}[1]{\expandafter\providecommand\csname #1\endcsname}%
\newcommand{\renewedcommand}[1]{\expandafter\providecommand\csname #1\endcsname{}%
  \expandafter\renewcommand\csname #1\endcsname}%
\newcommand{\newedenvironment}[1]{\newenvironment{#1}{}{}\renewenvironment{#1}}%
\let\newedcommand\renewedcommand
\let\renewedenvironment\newedenvironment
\makeatother
\let\mathon=$
\let\mathoff=$
\ifx\AtBeginDocument\undefined \newcommand{\AtBeginDocument}[1]{}\fi
\newbox\sizebox
\setlength{\hoffset}{0pt}\setlength{\voffset}{0pt}
\addtolength{\textheight}{\footskip}\setlength{\footskip}{0pt}
\addtolength{\textheight}{\topmargin}\setlength{\topmargin}{0pt}
\addtolength{\textheight}{\headheight}\setlength{\headheight}{0pt}
\addtolength{\textheight}{\headsep}\setlength{\headsep}{0pt}
\setlength{\textwidth}{349pt}
\newwrite\lthtmlwrite
\makeatletter
\let\realnormalsize=\normalsize
\global\topskip=2sp
\def\preveqno{}\let\real@float=\@float \let\realend@float=\end@float
\def\@float{\let\@savefreelist\@freelist\real@float}
\def\liih@math{\ifmmode$\else\bad@math\fi}
\def\end@float{\realend@float\global\let\@freelist\@savefreelist}
\let\real@dbflt=\@dbflt \let\end@dblfloat=\end@float
\let\@largefloatcheck=\relax
\let\if@boxedmulticols=\iftrue
\def\@dbflt{\let\@savefreelist\@freelist\real@dbflt}
\def\adjustnormalsize{\def\normalsize{\mathsurround=0pt \realnormalsize
 \parindent=0pt\abovedisplayskip=0pt\belowdisplayskip=0pt}%
 \def\phantompar{\csname par\endcsname}\normalsize}%
\def\lthtmltypeout#1{{\let\protect\string \immediate\write\lthtmlwrite{#1}}}%
\usepackage[tightpage,active]{preview}
\newbox\lthtmlPageBox
\newdimen\lthtmlCropMarkHeight
\newdimen\lthtmlCropMarkDepth
\long\def\lthtmlTightVBoxA#1{\def\lthtmllabel{#1}
    \setbox\lthtmlPageBox\vbox\bgroup\catcode`\_=8 }%
\long\def\lthtmlTightVBoxZ{\egroup
    \lthtmlCropMarkHeight=\ht\lthtmlPageBox \advance \lthtmlCropMarkHeight 6pt
    \lthtmlCropMarkDepth=\dp\lthtmlPageBox
    \lthtmltypeout{^^J:\lthtmllabel:lthtmlCropMarkHeight:=\the\lthtmlCropMarkHeight}%
    \lthtmltypeout{^^J:\lthtmllabel:lthtmlCropMarkDepth:=\the\lthtmlCropMarkDepth:1ex:=\the \dimexpr 1ex}%
    \begin{preview}\copy\lthtmlPageBox\end{preview}}%
\long\def\lthtmlTightFBoxA#1{\def\lthtmllabel{#1}%
    \adjustnormalsize\setbox\lthtmlPageBox=\vbox\bgroup %
    \let\ifinner=\iffalse \let\)\liih@math %
    \bgroup\catcode`\_=8 }%
\long\def\lthtmlTightFBoxZ{\egroup
    \@next\next\@currlist{}{\def\next{\voidb@x}}%
    \expandafter\box\next\egroup %
    \lthtmlCropMarkHeight=\ht\lthtmlPageBox \advance \lthtmlCropMarkHeight 6pt
    \lthtmlCropMarkDepth=\dp\lthtmlPageBox
    \lthtmltypeout{^^J:\lthtmllabel:lthtmlCropMarkHeight:=\the\lthtmlCropMarkHeight}%
    \lthtmltypeout{^^J:\lthtmllabel:lthtmlCropMarkDepth:=\the\lthtmlCropMarkDepth:1ex:=\the \dimexpr 1ex}%
    \begin{preview}\copy\lthtmlPageBox\end{preview}}%
    \long\def\lthtmlinlinemathA#1#2\lthtmlindisplaymathZ{\lthtmlTightVBoxA{#1}{\hbox\bgroup#2\egroup}\lthtmlTightVBoxZ}
    \def\lthtmlinlineA#1#2\lthtmlinlineZ{\lthtmlTightVBoxA{#1}{\hbox\bgroup#2\egroup}\lthtmlTightVBoxZ}
    \long\def\lthtmldisplayA#1#2\lthtmldisplayZ{\lthtmlTightVBoxA{#1}{#2}\lthtmlTightVBoxZ}
    \long\def\lthtmldisplayB#1#2\lthtmldisplayZ{\\edef\preveqno{(\theequation)}%
        \lthtmlTightVBoxA{#1}{\let\@eqnnum\relax#2}\lthtmlTightVBoxZ}
    \long\def\lthtmlfigureA#1{\let\@savefreelist\@freelist
        \lthtmlTightFBoxA{#1}}
    \long\def\lthtmlfigureZ{
        \lthtmlTightFBoxZ\global\let\@freelist\@savefreelist}
    \long\def\lthtmlpictureA#1{\let\@savefreelist\@freelist
        \lthtmlTightVBoxA{#1}}
    \long\def\lthtmlpictureZ{
        \lthtmlTightVBoxZ\global\let\@freelist\@savefreelist}
\def\lthtmlcheckvsize{\ifdim\ht\sizebox<\vsize 
  \ifdim\wd\sizebox<\hsize\expandafter\hfill\fi \expandafter\vfill
  \else\expandafter\vss\fi}%
\providecommand{\selectlanguage}[1]{}%
\makeatletter \tracingstats = 1 
\providecommand{\Rho}{\textrm{R}}
\providecommand{\Nu}{\textrm{N}}
\providecommand{\Alpha}{\textrm{A}}
\providecommand{\Eta}{\textrm{H}}
\providecommand{\Kappa}{\textrm{K}}
\providecommand{\Mu}{\textrm{M}}
\providecommand{\Epsilon}{\textrm{E}}
\providecommand{\omicron}{\textrm{o}}
\providecommand{\Beta}{\textrm{B}}
\providecommand{\Zeta}{\textrm{Z}}
\providecommand{\Tau}{\textrm{T}}
\providecommand{\Chi}{\textrm{X}}
\providecommand{\Omicron}{\textrm{O}}
\providecommand{\Iota}{\textrm{J}}


\begin{document}
\pagestyle{empty}\thispagestyle{empty}\lthtmltypeout{}%
\lthtmltypeout{latex2htmlLength hsize=\the\hsize}\lthtmltypeout{}%
\lthtmltypeout{latex2htmlLength vsize=\the\vsize}\lthtmltypeout{}%
\lthtmltypeout{latex2htmlLength hoffset=\the\hoffset}\lthtmltypeout{}%
\lthtmltypeout{latex2htmlLength voffset=\the\voffset}\lthtmltypeout{}%
\lthtmltypeout{latex2htmlLength topmargin=\the\topmargin}\lthtmltypeout{}%
\lthtmltypeout{latex2htmlLength topskip=\the\topskip}\lthtmltypeout{}%
\lthtmltypeout{latex2htmlLength headheight=\the\headheight}\lthtmltypeout{}%
\lthtmltypeout{latex2htmlLength headsep=\the\headsep}\lthtmltypeout{}%
\lthtmltypeout{latex2htmlLength parskip=\the\parskip}\lthtmltypeout{}%
\lthtmltypeout{latex2htmlLength oddsidemargin=\the\oddsidemargin}\lthtmltypeout{}%
\makeatletter
\if@twoside\lthtmltypeout{latex2htmlLength evensidemargin=\the\evensidemargin}%
\else\lthtmltypeout{latex2htmlLength evensidemargin=\the\oddsidemargin}\fi%
\lthtmltypeout{}%
\makeatother
\setcounter{page}{1}
\onecolumn

% !!! IMAGES START HERE !!!

\setcounter{secnumdepth}{3}
\stepcounter{chapter}
\stepcounter{section}
\stepcounter{section}
\stepcounter{section}
\stepcounter{section}
\stepcounter{section}
\stepcounter{section}
\stepcounter{chapter}
\stepcounter{section}
\stepcounter{subsection}
\stepcounter{subsection}
{\newpage\clearpage
\lthtmlinlinemathA{tex2html_wrap_inline22391}%
\bgroup\color{brown}$ \backslash$\egroup%
\lthtmlindisplaymathZ
\lthtmlcheckvsize\clearpage}

\stepcounter{section}
\stepcounter{subsection}
\stepcounter{subsubsection}
\stepcounter{subsubsection}
\stepcounter{subsubsection}
\stepcounter{subsubsection}
\stepcounter{subsection}
\stepcounter{subsubsection}
\stepcounter{paragraph}
\stepcounter{paragraph}
\stepcounter{paragraph}
\stepcounter{subsubsection}
{\newpage\clearpage
\lthtmlfigureA{figure431}%
\begin{figure}\begin{tabular}{ll}
\hline
\texttt{--enable-debug} & compile all projects with debug options set \\
\texttt{--enable-debug-symphony} & compile only SYMPHONY project with debug options \\
\texttt{--enable-msvc} & Under MSys2, compile so that executables do
not depend on the CYGWIN DLL \\
\texttt{--enable-static} & build static libraries \\
\texttt{--enable-static-executable} &  create a complete static executable \\
\texttt{--enable-gnu-packages} & compile with GNU packages \\
& compile interactive optimizer with readline library \\
\hline
\texttt{--disable-cgl-cuts} & disable generic cut generation \\
\texttt{--enable-sensitivity-analysis} & compile in the sensitivity analysis features \\
\texttt{--enable-root-only} & process only the root node \\
\texttt{--enable-frac-branching} & compile in the fractional branching option \\
\texttt{--enable-tests}&  perform additional sanity checks (for debugging purposes) \\
\texttt{--enable-tm-tests }& perform more tests  \\
\texttt{--enable-trace-path}&  additional debugging options \\
\texttt{--enable-cut-check}& additional debugging options \\
\texttt{--enable-statistics}& additional statistics \\
\texttt{--enable-pseudo-costs}& enable some experimental pseudo-cost branching tools \\
\texttt{--enable-draw-graph} &  enable IGD graph drawing application \\
\hline
\texttt{ --with-XXX-incdir} &  specify the directory with the header files for the XXX package \\
&where XXX is one of LP solver packages: cplex, glpk, osl, soplex, \\
& xpress \\
\texttt{--with-XXX-lib} &  specify the flags to link with the library  
XXX package \\
&where XXX is one of LP solver packages: cplex, glpk, osl, soplex, \\
& xpress \\
\texttt{--with-lp-solver[=lpsolver]} &  specify the LP solver in small 
letters (default lpsolver=clp) \\
\texttt{--with-application} &  compile the application library \\
\hline
\texttt{--enable-openmp} &   compile in OpenMP features \\
\texttt{--with-pvm } &  compile in parallel architecture (assuming that pvm is \\
&installed and the variable PVM\_ROOT is defined.) \\
\texttt{--without-cg} &  compile without cut generator module \\
\texttt{--without-cp} &  compile without cut pool module \\
\texttt{--without-lp} &  compile without LP solver module \\
\texttt{--without-tm} &  compile without tree manager module
\end{tabular}

\end{figure}%
\lthtmlfigureZ
\lthtmlcheckvsize\clearpage}

\stepcounter{subsubsection}
\stepcounter{paragraph}
\stepcounter{paragraph}
\stepcounter{subsubsection}
\stepcounter{subsection}
\stepcounter{subsubsection}
\stepcounter{subsubsection}
\stepcounter{subsubsection}
\stepcounter{subsubsection}
\stepcounter{paragraph}
\stepcounter{paragraph}
\stepcounter{paragraph}
\stepcounter{chapter}
\stepcounter{section}
\stepcounter{subsection}
\stepcounter{subsection}
\stepcounter{subsection}
\stepcounter{subsection}
\stepcounter{subsection}
\stepcounter{subsection}
\stepcounter{section}
\stepcounter{section}
\stepcounter{subsection}
\stepcounter{paragraph}
\stepcounter{paragraph}
\stepcounter{paragraph}
\stepcounter{paragraph}
\stepcounter{paragraph}
\stepcounter{paragraph}
\stepcounter{paragraph}
\stepcounter{paragraph}
{\newpage\clearpage
\lthtmlfigureA{figure800}%
\begin{figure}{\color{brown}
\begin{Verbatim}
[frame=lines]
int main(int argc, char **argv)
{
   sym_environment *env = sym_open_environment();
   sym_parse_command_line(env, argc, argv);
   sym_load_problem(env);
   sym_solve(env);
   sym_close_environment(env);
}\end{Verbatim}

}

\end{figure}%
\lthtmlfigureZ
\lthtmlcheckvsize\clearpage}

{\newpage\clearpage
\lthtmlfigureA{figure825}%
\begin{figure}{\color{brown}
\begin{Verbatim}
[frame=lines]
int main(int argc, char **argv)
{
   sym_environment *env = sym_open_environment();
   sym_parse_command_line(env, argc, argv);
   sym_load_problem(env);
   sym_set_int_param(env, "find_first_feasible", TRUE);
   sym_set_int_param(env, "node_selection_strategy", DEPTH_FIRST_SEARCH);
   sym_solve(env);
   sym_set_int_param(env, "find_first_feasible", FALSE);
   sym_set_int_param(env, "node_selection_strategy", BEST_FIRST_SEARCH);
   sym_warm_solve(env);
}\end{Verbatim}

}

\end{figure}%
\lthtmlfigureZ
\lthtmlcheckvsize\clearpage}

{\newpage\clearpage
\lthtmlfigureA{figure832}%
\begin{figure}{\color{brown}
\begin{Verbatim}
[frame=lines]
int main(int argc, char **argv)
{
   warm_start_desc *ws;
   sym_environment *env = sym_open_environment();
   sym_parse_command_line(env, argc, argv);
   sym_load_problem(env);
   sym_set_int_param(env, "node_limit", 100);
   sym_set_int_param(env, "keep_warm_start", TRUE);
   sym_solve(env);
   ws = sym_get_warm_start(env);
   sym_set_int_param(env, "node_limit", -1);
   sym_warm_solve(env);
   sym_set_obj_coeff(env, 0, 100);
   sym_set_obj_coeff(env, 200, 150);
   sym_set_warm_start(ws);
   sym_warm_solve(env);
}\end{Verbatim}

}

\end{figure}%
\lthtmlfigureZ
\lthtmlcheckvsize\clearpage}

{\newpage\clearpage
\lthtmlinlinemathA{tex2html_wrap_inline22516}%
\bgroup\color{brown}$ i$\egroup%
\lthtmlindisplaymathZ
\lthtmlcheckvsize\clearpage}

{\newpage\clearpage
\lthtmlinlinemathA{tex2html_wrap_inline22518}%
\bgroup\color{brown}$ i^\textrm{th}$\egroup%
\lthtmlindisplaymathZ
\lthtmlcheckvsize\clearpage}

{\newpage\clearpage
\lthtmlfigureA{figure840}%
\begin{figure}{\color{brown}
\begin{Verbatim}
[frame=lines]
int main(int argc, char **argv)
{
   sym_environment *env = sym_open_environment();
   sym_parse_command_line(env, argc, argv);
   sym_load_problem(env);
   sym_set_obj2_coeff(env, 0, 1);
   sym_mc_solve(env);
}\end{Verbatim}

}

\end{figure}%
\lthtmlfigureZ
\lthtmlcheckvsize\clearpage}

\stepcounter{subsection}
{\newpage\clearpage
\lthtmlfigureA{figure856}%
\begin{figure}{\color{brown}
\begin{Verbatim}
[frame=lines]
int main(int argc, char **argv)
{
   OsiSymSolverInterface si;
   si.parseCommandLine(argc, argv);
   si.loadProblem();
   si.branchAndBound();
}\end{Verbatim}

}

\end{figure}%
\lthtmlfigureZ
\lthtmlcheckvsize\clearpage}

\stepcounter{subsection}
\stepcounter{section}
\stepcounter{chapter}
\stepcounter{section}
{\newpage\clearpage
\lthtmlinlinemathA{tex2html_wrap_inline22531}%
\bgroup\color{brown}$ S$\egroup%
\lthtmlindisplaymathZ
\lthtmlcheckvsize\clearpage}

{\newpage\clearpage
\lthtmlinlinemathA{tex2html_wrap_inline22533}%
\bgroup\color{brown}$ c \in {\bf R}^S$\egroup%
\lthtmlindisplaymathZ
\lthtmlcheckvsize\clearpage}

{\newpage\clearpage
\lthtmlinlinemathA{tex2html_wrap_inline22535}%
\bgroup\color{brown}$ \hat{s} \in S$\egroup%
\lthtmlindisplaymathZ
\lthtmlcheckvsize\clearpage}

{\newpage\clearpage
\lthtmlinlinemathA{tex2html_wrap_inline22543}%
\bgroup\color{brown}$ \hat{s}$\egroup%
\lthtmlindisplaymathZ
\lthtmlcheckvsize\clearpage}

{\newpage\clearpage
\lthtmlinlinemathA{tex2html_wrap_inline22547}%
\bgroup\color{brown}$ n$\egroup%
\lthtmlindisplaymathZ
\lthtmlcheckvsize\clearpage}

{\newpage\clearpage
\lthtmlinlinemathA{tex2html_wrap_inline22551}%
\bgroup\color{brown}$ S_1, \ldots, S_n$\egroup%
\lthtmlindisplaymathZ
\lthtmlcheckvsize\clearpage}

{\newpage\clearpage
\lthtmlinlinemathA{tex2html_wrap_inline22553}%
\bgroup\color{brown}$ \cup_{i = 1}^n S_i = S$\egroup%
\lthtmlindisplaymathZ
\lthtmlcheckvsize\clearpage}

\stepcounter{section}
{\newpage\clearpage
\lthtmlpictureA{tex2html_wrap10512}%
\framebox[6.5in]{
\begin{minipage}{6.0in}
\vskip .1in
{\rm
{\bf Bounding Operation}\\
\underbar{Input:} A subproblem ${\cal S}$, described in
terms of a ``small'' set of inequalities ${\cal L'}$\  such that ${\cal
S} = \{x^s : s \in {\cal F}\;\hbox{\rm and}\;ax^s \leq \beta\;\forall
\;(a,\beta) \in {\cal L'}\}$\  and $\alpha$, an upper bound on the global
optimal value. \\
\underbar{Output:} Either (1) an optimal solution $s^* \in {\cal S}$\  to
the subproblem, (2) a lower bound on the optimal value of the
subproblem, or (3) a message {\tt pruned} indicating that the
subproblem should not be considered further. \\
{\bf Step 1.} Set ${\cal C} \leftarrow {\cal L'}$. \\
{\bf Step 2.} Solve the LP $\min\{cx : ax \leq \beta\;\forall\;(a, \beta)
\in {\cal C}\}$. \\
{\bf Step 3.} If the LP has a feasible solution $\hat{x}$, then go to
Step 4. Otherwise, STOP and output {\tt pruned}. This subproblem has no
feasible solutions. \\
{\bf Step 4.} If $c\hat{x} < \alpha$, then go to Step
5. Otherwise, STOP and output {\tt pruned}. This subproblem
cannot produce a solution of value better than $\alpha$. \\
{\bf Step 5.} If $\hat{x}$\  is the incidence vector of some $\hat{s}
\in {\cal S}$, then $\hat{s}$\  is the optimal solution to this
subproblem. STOP and output $\hat{s}$\  as $s^*$. Otherwise, apply
separation algorithms and heuristics to $\hat{x}$\  to get a set of
violated inequalities ${\cal C'}$. If ${\cal C'} = \emptyset$, then
$c\hat{x}$\  is a lower bound on the value of an optimal element of
${\cal S}$.  STOP and return $\hat{x}$\  and the lower bound
$c\hat{x}$. Otherwise, set ${\cal C} \leftarrow {\cal C} \cup {\cal
C'}$\  and go to Step 2.}
\end{minipage}
}%
\lthtmlpictureZ
\lthtmlcheckvsize\clearpage}

{\newpage\clearpage
\lthtmlinlinemathA{tex2html_wrap_inline22570}%
\bgroup\color{brown}$ \hbox{\em CP} =
(E, {\cal F})$\egroup%
\lthtmlindisplaymathZ
\lthtmlcheckvsize\clearpage}

{\newpage\clearpage
\lthtmlinlinemathA{tex2html_wrap_inline22572}%
\bgroup\color{brown}$ E$\egroup%
\lthtmlindisplaymathZ
\lthtmlcheckvsize\clearpage}

{\newpage\clearpage
\lthtmlinlinemathA{tex2html_wrap_inline22574}%
\bgroup\color{brown}$ {\cal F}
\subseteq 2^E$\egroup%
\lthtmlindisplaymathZ
\lthtmlcheckvsize\clearpage}

{\newpage\clearpage
\lthtmlinlinemathA{tex2html_wrap_inline22576}%
\bgroup\color{brown}$ c \in {\bf R}^E$\egroup%
\lthtmlindisplaymathZ
\lthtmlcheckvsize\clearpage}

{\newpage\clearpage
\lthtmlinlinemathA{tex2html_wrap_inline22578}%
\bgroup\color{brown}$ {\cal F}$\egroup%
\lthtmlindisplaymathZ
\lthtmlcheckvsize\clearpage}

{\newpage\clearpage
\lthtmlinlinemathA{tex2html_wrap_inline22580}%
\bgroup\color{brown}$ {\cal
P}$\egroup%
\lthtmlindisplaymathZ
\lthtmlcheckvsize\clearpage}

{\newpage\clearpage
\lthtmlinlinemathA{tex2html_wrap_inline22584}%
\bgroup\color{brown}$ {\cal L}$\egroup%
\lthtmlindisplaymathZ
\lthtmlcheckvsize\clearpage}

{\newpage\clearpage
\lthtmlinlinemathA{tex2html_wrap_indisplay22588}%
$\displaystyle {\cal P} = \{x \in {\bf R}^n: ax \leq \beta\;\;\forall\;(a, \beta) \in
{\cal L}\}.$%
\lthtmlindisplaymathZ
\lthtmlcheckvsize\clearpage}

{\newpage\clearpage
\lthtmlpictureA{tex2html_wrap10526}%
\framebox[6.5in]{
\begin{minipage}{6.0in}
\vskip .1in
{\rm
{\bf Branching Operation} \\
\underbar{Input:} A subproblem ${\cal S}$\  and $\hat{x}$, the LP solution
yielding the lower bound. \\
\underbar{Output:} $S_1, \ldots, S_p$\  such that ${\cal S} = \cup_{i = 1}^p
S_i$. \\
{\bf Step 1.} Determine sets ${\cal L}_1, \ldots, {\cal L}_p$\  of
inequalities such that ${\cal S} = \cup_{i = 1}^n \{x \in {\cal S}: ax \leq
\beta\;\forall\;(a, \beta) \in {\cal L}_i\}$\  and $\hat{x} \notin
\cup_{i = 1}^n S_i$. \\
{\bf Step 2.} Set $S_i = \{x \in {\cal S}: ax \leq
\beta\;\;\forall\;(a, \beta) \in {\cal L}_i \cup {\cal L}'\}$\  where
${\cal L'}$\  is the set of inequalities used to describe ${\cal S}$.}
\end{minipage}
}%
\lthtmlpictureZ
\lthtmlcheckvsize\clearpage}

{\newpage\clearpage
\lthtmlpictureA{tex2html_wrap10532}%
\framebox[6.5in]{
\begin{minipage}{6.0in}
\vskip .1in
{\rm
{\bf Generic Branch and Cut Algorithm}\\
\underbar{Input:} A data array specifying the problem instance.\\
\underbar{Output:} The global optimal solution $s^*$\  to the problem
instance. \\
{\bf Step 1.} Generate a ``good'' feasible solution ${\hat s}$\  using
heuristics. Set $\alpha \leftarrow c(\hat{s})$. \\
{\bf Step 2.} Generate the first subproblem ${\cal S}^I$\  by constructing a
small set ${\cal L'}$\  of inequalities valid for ${\cal P}$. Set $A
\leftarrow \{{\cal S}^I\}$. \\
{\bf Step 3.} If $A = \emptyset$, STOP and output $\hat{s}$\  as the
global optimum $s^*$. Otherwise, choose some ${\cal S} \in A$. Set $A
\leftarrow A \setminus \{{\cal S}\}$. Process ${\cal S}$. \\
{\bf Step 4.} If the result of Step 3 is a feasible solution
$\overline{s}$, then $c\overline{s} < c\hat{s}$.
Set $\hat{s} \leftarrow \overline{s}$\  and $\alpha \leftarrow
c(\overline{s})$\  and go to Step 3. If the subproblem was pruned, go to
Step 3. Otherwise, go to Step 5. \\
{\bf Step 5.} Perform the branching operation. Add the set of
subproblems generated to $A$\  and go to Step 3.}
\end{minipage}
}%
\lthtmlpictureZ
\lthtmlcheckvsize\clearpage}

\stepcounter{section}
\stepcounter{subsection}
\stepcounter{subsection}
\stepcounter{subsubsection}
\stepcounter{subsubsection}
\stepcounter{subsubsection}
\stepcounter{subsection}
\stepcounter{subsubsection}
\stepcounter{subsubsection}
\stepcounter{subsubsection}
\stepcounter{subsubsection}
\stepcounter{subsubsection}
\stepcounter{subsection}
\stepcounter{section}
\stepcounter{subsection}
\stepcounter{subsubsection}
\stepcounter{subsubsection}
\stepcounter{paragraph}
\stepcounter{paragraph}
\stepcounter{subsubsection}
{\newpage\clearpage
\lthtmldisplayA{displaymath22627}%
\begin{displaymath}\begin{array}{lrcl}
\mathop{\text{vmin}}& [cx, dx],\\
\textrm{s.t.} & Ax & \leq & b, \\
& x & \in & \mathbb{Z}^{n}.
\end{array}\end{displaymath}%
\lthtmldisplayZ
\lthtmlcheckvsize\clearpage}

{\newpage\clearpage
\lthtmlinlinemathA{tex2html_wrap_inline22629}%
\bgroup\color{brown}$ p$\egroup%
\lthtmlindisplaymathZ
\lthtmlcheckvsize\clearpage}

{\newpage\clearpage
\lthtmlinlinemathA{tex2html_wrap_inline22631}%
\bgroup\color{brown}$ q$\egroup%
\lthtmlindisplaymathZ
\lthtmlcheckvsize\clearpage}

{\newpage\clearpage
\lthtmlinlinemathA{tex2html_wrap_inline22633}%
\bgroup\color{brown}$ cq \leq cp$\egroup%
\lthtmlindisplaymathZ
\lthtmlcheckvsize\clearpage}

{\newpage\clearpage
\lthtmlinlinemathA{tex2html_wrap_inline22635}%
\bgroup\color{brown}$ dq \leq dp$\egroup%
\lthtmlindisplaymathZ
\lthtmlcheckvsize\clearpage}

{\newpage\clearpage
\lthtmlinlinemathA{tex2html_wrap_inline22637}%
\bgroup\color{brown}$ 0 \leq \alpha \leq 1$\egroup%
\lthtmlindisplaymathZ
\lthtmlcheckvsize\clearpage}

{\newpage\clearpage
\lthtmlinlinemathA{tex2html_wrap_indisplay22639}%
$\displaystyle (\alpha c + (1 - \alpha) d) x.$%
\lthtmlindisplaymathZ
\lthtmlcheckvsize\clearpage}

{\newpage\clearpage
\lthtmlinlinemathA{tex2html_wrap_inline22641}%
\bgroup\color{brown}$ \alpha$\egroup%
\lthtmlindisplaymathZ
\lthtmlcheckvsize\clearpage}

{\newpage\clearpage
\lthtmlinlinemathA{tex2html_wrap_inline22643}%
\bgroup\color{brown}$ x^c$\egroup%
\lthtmlindisplaymathZ
\lthtmlcheckvsize\clearpage}

{\newpage\clearpage
\lthtmlinlinemathA{tex2html_wrap_inline22645}%
\bgroup\color{brown}$ \alpha = 1$\egroup%
\lthtmlindisplaymathZ
\lthtmlcheckvsize\clearpage}

{\newpage\clearpage
\lthtmlinlinemathA{tex2html_wrap_inline22647}%
\bgroup\color{brown}$ x^d$\egroup%
\lthtmlindisplaymathZ
\lthtmlcheckvsize\clearpage}

{\newpage\clearpage
\lthtmlinlinemathA{tex2html_wrap_inline22649}%
\bgroup\color{brown}$ \alpha = 0$\egroup%
\lthtmlindisplaymathZ
\lthtmlcheckvsize\clearpage}

{\newpage\clearpage
\lthtmlinlinemathA{tex2html_wrap_indisplay22653}%
$\displaystyle \max \{\alpha (cp - cx^c), (1 - \alpha)(dp - dx^d)\}.$%
\lthtmlindisplaymathZ
\lthtmlcheckvsize\clearpage}

\stepcounter{subsection}
\stepcounter{subsubsection}
\stepcounter{subsubsection}
\stepcounter{subsubsection}
\stepcounter{subsection}
\stepcounter{subsubsection}
\stepcounter{subsubsection}
\stepcounter{subsubsection}
\stepcounter{subsection}
\stepcounter{subsection}
\stepcounter{subsubsection}
\stepcounter{subsubsection}
\stepcounter{section}
\stepcounter{subsection}
\stepcounter{subsection}
\stepcounter{subsection}
\stepcounter{chapter}
\stepcounter{section}
\stepcounter{section}
\stepcounter{subsection}
\stepcounter{subsection}
\stepcounter{subsubsection}
\stepcounter{subsubsection}
\stepcounter{section}
\stepcounter{section}
\stepcounter{section}
\stepcounter{subsection}
\stepcounter{subsection}
\stepcounter{section}
\stepcounter{subsection}
\stepcounter{subsection}
\stepcounter{subsection}
\stepcounter{subsection}
\stepcounter{subsection}
\stepcounter{section}
{\newpage\clearpage
\lthtmlinlinemathA{tex2html_wrap_inline22720}%
\bgroup\color{brown}$ (i,j)$\egroup%
\lthtmlindisplaymathZ
\lthtmlcheckvsize\clearpage}

{\newpage\clearpage
\lthtmlinlinemathA{tex2html_wrap_inline22722}%
\bgroup\color{brown}$ i<j$\egroup%
\lthtmlindisplaymathZ
\lthtmlcheckvsize\clearpage}

\stepcounter{chapter}
\stepcounter{section}
\stepcounter{subsection}
\stepcounter{subsubsection}
\stepcounter{subsubsection}
\stepcounter{subsubsection}
\stepcounter{subsubsection}
\stepcounter{subsubsection}
\stepcounter{subsubsection}
{\newpage\clearpage
\lthtmlinlinemathA{tex2html_wrap_inline22808}%
$ \leq$%
\lthtmlindisplaymathZ
\lthtmlcheckvsize\clearpage}

{\newpage\clearpage
\lthtmlinlinemathA{tex2html_wrap_inline22810}%
$ \geq$%
\lthtmlindisplaymathZ
\lthtmlcheckvsize\clearpage}

\stepcounter{subsubsection}
\stepcounter{subsubsection}
\stepcounter{subsubsection}
\stepcounter{subsubsection}
\stepcounter{subsubsection}
\stepcounter{subsubsection}
\stepcounter{subsubsection}
\stepcounter{subsubsection}
\stepcounter{subsubsection}
\stepcounter{subsection}
\stepcounter{subsubsection}
\stepcounter{subsubsection}
\stepcounter{subsubsection}
\stepcounter{subsubsection}
\stepcounter{subsubsection}
\stepcounter{subsubsection}
\stepcounter{subsubsection}
\stepcounter{subsection}
\stepcounter{subsubsection}
\stepcounter{subsubsection}
\stepcounter{subsubsection}
\stepcounter{subsubsection}
\stepcounter{subsubsection}
\stepcounter{subsubsection}
\stepcounter{subsubsection}
\stepcounter{subsection}
\stepcounter{subsubsection}
\stepcounter{subsubsection}
\stepcounter{subsubsection}
\stepcounter{subsubsection}
\stepcounter{subsubsection}
\stepcounter{subsubsection}
\stepcounter{subsubsection}
\stepcounter{subsubsection}
\stepcounter{subsubsection}
\stepcounter{subsubsection}
\stepcounter{subsubsection}
\stepcounter{subsubsection}
\stepcounter{subsubsection}
\stepcounter{subsubsection}
\stepcounter{subsubsection}
\stepcounter{subsubsection}
\stepcounter{subsubsection}
\stepcounter{subsubsection}
\stepcounter{subsubsection}
\stepcounter{subsubsection}
\stepcounter{subsubsection}
\stepcounter{subsubsection}
\stepcounter{subsubsection}
\stepcounter{subsubsection}
\stepcounter{subsubsection}
\stepcounter{subsubsection}
\stepcounter{subsection}
\stepcounter{subsubsection}
\stepcounter{subsubsection}
\stepcounter{subsubsection}
\stepcounter{subsubsection}
\stepcounter{subsubsection}
\stepcounter{subsubsection}
\stepcounter{subsubsection}
\stepcounter{subsubsection}
\stepcounter{subsubsection}
\stepcounter{subsubsection}
\stepcounter{subsubsection}
\stepcounter{subsubsection}
\stepcounter{subsubsection}
\stepcounter{subsubsection}
\stepcounter{subsubsection}
\stepcounter{subsubsection}
\stepcounter{subsubsection}
\stepcounter{subsection}
\stepcounter{subsubsection}
\stepcounter{subsubsection}
\stepcounter{subsubsection}
\stepcounter{subsubsection}
\stepcounter{subsubsection}
\stepcounter{subsubsection}
\stepcounter{subsection}
\stepcounter{subsubsection}
\stepcounter{subsubsection}
\stepcounter{subsubsection}
\stepcounter{subsubsection}
\stepcounter{section}
{\newpage\clearpage
\lthtmlpictureA{tex2html_wrap13886}%
\resizebox{15cm}{12cm}{
\begin{tabular}{|l||l||l|} \hline
{\bf C++ Interface} & {\bf C Interface} & {\bf Description}\\\hline \hline
OsiSymSolverInterface & sym\_open\_environment &
create a new environment.\\\hline \hline
loadProblem & sym\_load\_problem &
load the problem read trough an MPS or GMPL file\\\hline \hline
branchAndBound & sym\_solve/sym\_warm\_solve &
solve the MILP problem from scratch or \\
& & from a warm start if loaded. \\\hline \hline
resolve & sym\_warm\_solve &
re-solve the MILP problem after some modifications.\\\hline \hline
initialSolve & sym\_solve &
solve the MILP problem from scratch.\\\hline \hline
multiCriteriaBranchAndBound & sym\_mc\_solve &
solve the multi criteria problem.\\\hline \hline
setInitialData & sym\_set\_defaults &
set the parameters to their defaults.\\\hline \hline
parseCommandLine & sym\_parse\_command\_line &
read the command line arguments.\\\hline \hline
findInitialBounds & sym\_find\_initial\_bounds &
find the initial bounds via the user defined heuristics.\\\hline \hline
createPermanentCutPools & sym\_create\_permanent\_cut\_pools &
save the global cuts. \\\hline \hline
loadProblem & sym\_explicit\_load\_problem &
load the problem through a set of arrays. \\\hline \hline
getWarmStart & sym\_get\_warm\_start &
get the warm start description.\\\hline \hline
setWarmStart & sym\_set\_warm\_start &
set the warm start description. \\\hline
getLbForNewRhs & sym\_get\_lb\_for\_new\_rhs &
find a lower bound to the new rhs problem\\
&&using the post solution info.\\\hline \hline
getUbForNewRhs & sym\_get\_lb\_for\_new\_rhs &
find an upper bound to the new rhs problem.\\
&&using the post solution info.\\\hline \hline
getLbForNewObj & sym\_get\_lb\_for\_new\_rhs &
find a lower bound to the new obj problem.\\
&&using the post solution info.\\\hline \hline
getUbForNewObj & sym\_get\_lb\_for\_new\_rhs &
find an upper bound to the new obj problem.\\
&&using the post solution info.\\\hline \hline
reset & sym\_close\_environment &
return the allocated memory.\\\hline \hline
setIntParam & sym\_set\_int\_param &
set the integer type OSI parameter.\\\hline \hline
setSymParam(int) & sym\_set\_int\_param &
set the integer type SYMPHONY parameter.\\\hline \hline
setDblParam & sym\_set\_dbl\_param &
set the double type OSI parameter.\\\hline \hline
setSymParam(double) & sym\_set\_dbl\_param &
set the double type SYMPHONY parameter.\\\hline \hline
setStrParam & sym\_set\_str\_param &
set the string type OSI parameter.\\\hline \hline
setSymParam(string) & sym\_set\_str\_param &
set the string type SYMPHONY parameter.\\\hline \hline
getIntParam & sym\_get\_int\_param &
get the value of the integer type OSI parameter. \\\hline \hline
getSymParam(int \&) & sym\_get\_int\_param &
get the value of the integer type SYMPHONY parameter. \\\hline \hline
getDblParam & sym\_get\_dbl\_param &
get the value of the double type OSI parameter. \\\hline \hline
getSymParam(double \&) & sym\_get\_dbl\_param &
get the value of the double type SYMPHONY parameter. \\\hline \hline
getStrParam & sym\_get\_str\_param &
get the value of the string type OSI parameter. \\\hline \hline
getSymParam(string \&) & sym\_get\_str\_param &
get the value of the string type SYMPHONY parameter. \\\hline \hline
isProvenOptimal & sym\_is\_proven\_optimal &
query the problem status. \\\hline \hline
isProvenPrimalInfeasible & sym\_is\_proven\_primal\_infeasible &
query the problem status. \\\hline \hline
isPrimalObjectiveLimitReached & sym\_is\_target\_gap\_achieved &
query the problem status. \\\hline \hline
isIterationLimitReached & sym\_is\_iteration\_limit\_reached &
query the problem status. \\\hline \hline
isTimeLimitReached & sym\_is\_time\_limit\_reached &
query the problem status. \\\hline \hline
isTargetGapReached & sym\_is\_target\_gap\_achieved &
query the problem status. \\\hline \hline
getNumCols & sym\_get\_num\_cols &
get the number of columns. \\\hline \hline
getNumRows & sym\_get\_num\_rows &
get the number of rows. \\\hline \hline
getNumElements & sym\_get\_num\_elements &
get the number of nonzero elements. \\\hline \hline
getColLower & sym\_get\_col\_lower &
get the column lower bounds. \\\hline \hline
getColUpper & sym\_get\_col\_upper &
get the column upper bounds. \\\hline \hline
getRowSense & sym\_get\_row\_sense &
get the row senses. \\\hline \hline
getRightHandSide & sym\_get\_rhs &
get the rhs values. \\\hline \hline
getRowRange & sym\_get\_row\_range &
get the row range values. \\\hline \hline
getRowLower & sym\_get\_row\_lower &
get the row lower bounds. \\\hline \hline
getRowUpper & sym\_get\_row\_upper &
get the row upper bounds. \\\hline \hline
getObjCoefficients & sym\_get\_obj\_coeff &
get the objective function vector. \\\hline
\end{tabular}
}%
\lthtmlpictureZ
\lthtmlcheckvsize\clearpage}

{\newpage\clearpage
\lthtmlpictureA{tex2html_wrap13892}%
\resizebox{15cm}{10.5cm}{
\begin{tabular}{|l||l||l|} \hline
{\bf C++ Interface} & {\bf C Interface} & {\bf Description}\\\hline \hline
getObjSense & sym\_get\_obj\_sense &
get the objective sense. \\\hline \hline
isContinuous & sym\_is\_continuous &
query the variable type.\\\hline \hline
isBinary & sym\_is\_binary &
query the variable type.\\\hline \hline
isInteger & sym\_is\_integer &
query the variable type.\\\hline \hline
isIntegerNonBinary & - &
query the variable type.\\\hline \hline
isFreeBinary & sym\_is\_binary &
query the variable type.\\\hline \hline
getMatrixByRow & - &
get the constraint matrix by row oriented. \\\hline \hline
getMatrixByCol & - &
get the constraint matrix by column oriented. \\\hline \hline
getInfinity & - &
get the infinity definition of SYMPHONY. \\\hline \hline
getColSolution & sym\_get\_col\_solution &
get the current best column solution. \\\hline \hline
getRowActivity & sym\_get\_row\_activity &
get the current row activity. \\\hline \hline
getObjValue & sym\_get\_obj\_val &
get the current best objective value. \\\hline \hline
getPrimalBound & sym\_get\_primal\_bound &
get the primal upper bound. \\\hline \hline
getIterationCount & sym\_get\_iteration\_count &
get the number of the analyzed tree nodes. \\\hline \hline
setObjCoeff & sym\_set\_obj\_coeff &
set the objective function vector. \\\hline \hline
setObj2Coeff & sym\_set\_obj2\_coeff &
set the second objective function vector. \\\hline \hline
setColLower & sym\_set\_col\_lower &
set the column lower bounds. \\\hline \hline
setColUpper & sym\_set\_col\_upper &
set the column upper bounds. \\\hline \hline
setRowLower & sym\_set\_row\_lower &
set the row lower bounds. \\\hline \hline
setRowUpper & sym\_set\_row\_upper &
set the row upper bounds. \\\hline \hline
setRowType & sym\_set\_row\_type &
set the row characteristics. \\\hline \hline
setObjSense & sym\_set\_obj\_sense &
set the objective sense. \\\hline \hline
setColSolution & sym\_set\_col\_solution &
set the current solution. \\\hline \hline
setContinuous & sym\_set\_continuous &
set the variable type. \\\hline \hline
setInteger & sym\_set\_integer &
set the variable type. \\\hline \hline
setColName & sym\_set\_col\_names &
set the column names. \\\hline \hline
addCol & sym\_add\_col &
add columns to the constraint matrix. \\\hline \hline
addRow & sym\_add\_row &
add rows to the constraint matrix. \\\hline \hline
deleteCols & sym\_delete\_cols &
delete some columns from the constraint matrix. \\\hline \hline
deleteRows & sym\_delete\_rows &
delete some rows from the constraint matrix. \\\hline \hline
writeMps & - &
write the current problem in MPS format. \\\hline \hline
applyRowCut & - &
add some row cuts. \\\hline \hline
applyColCut & - &
add some column cuts. \\\hline \hline
SymWarmStart(warm\_start\_desc *) & sym\_create\_copy\_warm\_start &
create a SYMPHONY warm start by copying the given one. \\\hline \hline
SymWarmStart(char *) & sym\_read\_warm\_start &
create a SYMPHONY warm start reading from file. \\\hline \hline
getCopyOfWarmStartDesc & sym\_create\_copy\_warm\_start &
get the copy of the warm start structure. \\\hline \hline
writeToFile & sym\_write\_warm\_start\_desc &
write the loaded warm start to a file. \\\hline
\end{tabular}
}%
\lthtmlpictureZ
\lthtmlcheckvsize\clearpage}

\stepcounter{section}
\stepcounter{subsection}
\stepcounter{subsubsection}
\stepcounter{subsubsection}
\stepcounter{subsubsection}
\stepcounter{subsubsection}
\stepcounter{subsubsection}
\stepcounter{subsubsection}
\stepcounter{subsubsection}
\stepcounter{subsubsection}
\stepcounter{subsubsection}
\stepcounter{subsubsection}
\stepcounter{subsubsection}
\stepcounter{subsubsection}
\stepcounter{subsubsection}
\stepcounter{subsubsection}
\stepcounter{subsubsection}
\stepcounter{subsection}
\stepcounter{subsubsection}
\stepcounter{paragraph}
\stepcounter{subsubsection}
{\newpage\clearpage
\lthtmlinlinemathA{tex2html_wrap_inline24419}%
\bgroup\color{brown}$ \times$\egroup%
\lthtmlindisplaymathZ
\lthtmlcheckvsize\clearpage}

{\newpage\clearpage
\lthtmlinlinemathA{tex2html_wrap_inline24421}%
\bgroup\color{brown}$ \leq$\egroup%
\lthtmlindisplaymathZ
\lthtmlcheckvsize\clearpage}

{\newpage\clearpage
\lthtmlinlinemathA{tex2html_wrap_inline24423}%
\bgroup\color{brown}$ =$\egroup%
\lthtmlindisplaymathZ
\lthtmlcheckvsize\clearpage}

{\newpage\clearpage
\lthtmlinlinemathA{tex2html_wrap_inline24425}%
\bgroup\color{brown}$ \geq$\egroup%
\lthtmlindisplaymathZ
\lthtmlcheckvsize\clearpage}

{\newpage\clearpage
\lthtmlinlinemathA{tex2html_wrap_inline24427}%
\bgroup\color{brown}$ 0^{th}$\egroup%
\lthtmlindisplaymathZ
\lthtmlcheckvsize\clearpage}

{\newpage\clearpage
\lthtmlinlinemathA{tex2html_wrap_inline24429}%
\bgroup\color{brown}$ 1^{st}$\egroup%
\lthtmlindisplaymathZ
\lthtmlcheckvsize\clearpage}

\stepcounter{paragraph}
\stepcounter{subsubsection}
{\newpage\clearpage
\lthtmlinlinemathA{tex2html_wrap_inline24433}%
$ {\tt rhs}+{\tt range}$%
\lthtmlindisplaymathZ
\lthtmlcheckvsize\clearpage}

{\newpage\clearpage
\lthtmlinlinemathA{tex2html_wrap_inline24437}%
$ =$%
\lthtmlindisplaymathZ
\lthtmlcheckvsize\clearpage}

\stepcounter{paragraph}
\stepcounter{subsubsection}
\stepcounter{paragraph}
\stepcounter{subsubsection}
\stepcounter{subsubsection}
\stepcounter{subsubsection}
\stepcounter{subsubsection}
\stepcounter{subsubsection}
{\newpage\clearpage
\lthtmlinlinemathA{tex2html_wrap_inline24506}%
$ \epsilon$%
\lthtmlindisplaymathZ
\lthtmlcheckvsize\clearpage}

\stepcounter{subsubsection}
\stepcounter{subsubsection}
\stepcounter{subsubsection}
\stepcounter{subsubsection}
\stepcounter{subsubsection}
\stepcounter{subsubsection}
\stepcounter{subsubsection}
\stepcounter{subsubsection}
\stepcounter{subsubsection}
\stepcounter{subsubsection}
\stepcounter{subsubsection}
\stepcounter{subsubsection}
\stepcounter{subsubsection}
{\newpage\clearpage
\lthtmlinlinemathA{tex2html_wrap_inline25022}%
$ -1$%
\lthtmlindisplaymathZ
\lthtmlcheckvsize\clearpage}

\stepcounter{subsubsection}
\stepcounter{subsubsection}
\stepcounter{subsubsection}
{\newpage\clearpage
\lthtmlinlinemathA{tex2html_wrap_inline25149}%
$ ^{\tt th}$%
\lthtmlindisplaymathZ
\lthtmlcheckvsize\clearpage}

\stepcounter{subsubsection}
\stepcounter{subsection}
\stepcounter{subsubsection}
\stepcounter{subsubsection}
\stepcounter{subsubsection}
\stepcounter{subsubsection}
\stepcounter{subsubsection}
\stepcounter{subsection}
\stepcounter{subsubsection}
\stepcounter{subsubsection}
\stepcounter{subsubsection}
\stepcounter{subsubsection}
\stepcounter{subsubsection}
\stepcounter{subsubsection}
\stepcounter{subsection}
\stepcounter{subsubsection}
\stepcounter{subsubsection}
\stepcounter{subsubsection}
\stepcounter{subsubsection}
\stepcounter{section}
\stepcounter{subsection}
\stepcounter{subsection}
{\newpage\clearpage
\lthtmlinlinemathA{tex2html_wrap_inline25519}%
$ 2$%
\lthtmlindisplaymathZ
\lthtmlcheckvsize\clearpage}

{\newpage\clearpage
\lthtmlinlinemathA{tex2html_wrap_inline25521}%
$ 5$%
\lthtmlindisplaymathZ
\lthtmlcheckvsize\clearpage}

{\newpage\clearpage
\lthtmlinlinemathA{tex2html_wrap_inline25523}%
$ 1$%
\lthtmlindisplaymathZ
\lthtmlcheckvsize\clearpage}

\stepcounter{subsection}
\stepcounter{subsection}
\stepcounter{subsection}
{\newpage\clearpage
\lthtmlinlinemathA{tex2html_wrap_inline25650}%
$ z_i^+,
z_i^-$%
\lthtmlindisplaymathZ
\lthtmlcheckvsize\clearpage}

{\newpage\clearpage
\lthtmlinlinemathA{tex2html_wrap_inline25652}%
$ i$%
\lthtmlindisplaymathZ
\lthtmlcheckvsize\clearpage}

{\newpage\clearpage
\lthtmlinlinemathA{tex2html_wrap_inline25656}%
$ s_i =
\alpha\times\min\{z_i^+, z_i^-\} + (1-\alpha)\times\max\{z_i^+,z_i^-\}$%
\lthtmlindisplaymathZ
\lthtmlcheckvsize\clearpage}

{\newpage\clearpage
\lthtmlinlinemathA{tex2html_wrap_inline25658}%
$ \alpha$%
\lthtmlindisplaymathZ
\lthtmlcheckvsize\clearpage}

{\newpage\clearpage
\lthtmlinlinemathA{tex2html_wrap_inline25660}%
$ [0,1]$%
\lthtmlindisplaymathZ
\lthtmlcheckvsize\clearpage}

{\newpage\clearpage
\lthtmlinlinemathA{tex2html_wrap_inline25662}%
$ n^{\textrm th}$%
\lthtmlindisplaymathZ
\lthtmlcheckvsize\clearpage}

{\newpage\clearpage
\lthtmlinlinemathA{tex2html_wrap_inline25664}%
$ n$%
\lthtmlindisplaymathZ
\lthtmlcheckvsize\clearpage}

\stepcounter{subsection}
\stepcounter{subsection}
\stepcounter{subsection}
{\newpage\clearpage
\lthtmlpictureA{tex2html_wrap18432}%
\resizebox{16cm}{7cm}{
\begin{tabular}{|l||l||l|} \hline
{\bf C++ Interface} & {\bf C Interface} & {\bf Value}\\\hline \hline
OsiSymVerbosity & verbosity & -user defined- \\
\hline \hline
OsiSymWarmStart & warm\_start & -boolean- \\
\hline \hline
OsiSymNodeLimit &  & \\
OsiMaxNumIteration & node\_limit & -user defined-\\
OsiMaxNumIterationHotStart & & \\
\hline \hline
OsiSymFindFirstFeasible & find\_first\_feasible & -boolean- \\
\hline \hline
OsiSymSearchStrategy & node\_selection\_rule & LOWEST\_LP\_FIRST \\
& & HIGHEST\_LP\_FIRST \\
& & BREADTH\_FIRST\_SEARCH \\
& & DEPTH\_FIRST\_SEARCH \\
\hline \hline
OsiSymUsePermanentCutPools & use\_permanent\_cut\_pools & -boolean- \\
\hline \hline
OsiSymGenerateCglGomoryCuts & generate\_cgl\_gomory\_cuts & -boolean- \\
\hline \hline
OsiSymGenerateCglKnapsackCuts & generate\_cgl\_knapsack\_cuts & -boolean- \\
\hline \hline
OsiSymGenerateCglOddHoleCuts & generate\_cgl\_oddhole\_cuts & -boolean- \\
\hline \hline
OsiSymGenerateCglProbingCuts & generate\_cgl\_probing\_cuts & -boolean- \\
\hline \hline
OsiSymGenerateCglCliqueCuts & generate\_cgl\_clique\_cuts & -boolean- \\
\hline \hline
OsiSymGenerateCglFlowAndCoverCuts & generate\_cgl\_flow\_and\_cover\_cuts & -boolean- \\
\hline \hline
OsiSymGenerateCglRoundingCuts & generate\_cgl\_rounding\_cuts & -boolean- \\
\hline \hline
OsiSymGenerateCglLiftAndProjectCuts & generate\_cgl\_lift\_and\_project\_cuts & -boolean- \\
\hline \hline
OsiSymKeepWarmStart & keep\_warm\_start & -boolean- \\
\hline \hline
OsiSymTrimWarmTree & trim\_warm\_tree * -boolean- \\
\hline \hline
OsiSymDoReducedCostFixing & do\_reduced\_cost\_fixing & -boolean- \\
\hline \hline
OsiSymMCFindSupportedSolutions &
mc\_find\_supported\_solutions & -boolean- \\
\hline \hline
OsiSymSensitivityAnalysis & sensitivity\_analysis & -boolean- \\
\hline \hline
OsiSymRandomSeed & random\_seed & -user defined-\\
\hline \hline
OsiSymDivingStrategy & diving\_strategy & BEST\_ESTIMATE \\
& & COMP\_BEST\_K \\
& & COMP\_BEST\_K\_GAP \\
\hline \hline
OsiSymDivingK & diving\_k & -user defined- \\
\hline \hline
OsiSymDivingThreshold & diving\_threshold & -user defined- \\
\hline \hline
OsiSymGranularity & granularity & -user defined- \\
\hline \hline
OsiSymTimeLimit & time\_limit & -user defined- \\
\hline \hline
OsiSymGapLimit & gap\_limit & -user defined- \\
\hline \hline
OsiObjOffset & - & -user defined- \\
\hline \hline
OsiProbName & problem\_name & -user defined- \\
\hline
\end{tabular}
}%
\lthtmlpictureZ
\lthtmlcheckvsize\clearpage}


\end{document}
